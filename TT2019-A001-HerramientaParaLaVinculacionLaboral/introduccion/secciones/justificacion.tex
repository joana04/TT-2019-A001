\section{Justificación}

Hoy día la tasa de ninis en México esta incrementado llegando a una cifra de 3.9 millones de personas según el \cite{INEGI} \cite{OECD2}, generando un costo anual de 194,000 millones de pesos.\\

Como se mencionó, la educación y el desempleo, son dos grandes contribuyentes para la formación de un NINI, ya que constantemente no cuentan con los recursos para continuar o concluir su formación académica, generando un impacto considerable en la búsqueda de empleo.\\

Las bolsas de trabajo actuales despliegan las vacantes basándose solamente en la información ingresada por el solicitante sin realizar un estudio completo sobre el mismo y/o las oportunidades disponibles, además de que el radio de búsqueda de las vacantes son estatales en lugar de tomar en cuenta un radio especificado por el usuario, algunos ejemplos como OCCMUNDIAL que toma en cuenta el salario, la fecha de contratación, ubicación y el tipo de contratación \cite{OCC_Blog}, el portal  gubernamental gob.mx/empleo  filtra solamente por carrera u oficio y la entidad federativa \cite{OCC_Busqueda}, entre otras herramientas. \\
 
 En nuestro proyecto, definiremos un marco de comparación para reportar la eficiencia que se puede llegar a obtener con nuestro enfoque de trabajo. 
El desarrollo de la herramienta contempla servicios web, además de la implementación de algoritmos de mapeo geográfico, bases de datos, así como algoritmos de selección que serán de utilidad para estudiar y filtrar las opciones disponibles tanto para el solicitante como para las instituciones. \\
 

 \bigskip


%%Los factores que más inciden en la condición de convertirse en nini están: IDH del municipio de residencia del individuo, sexo, edad, número de ocupados en el hogar y, en menor medida, jefatura masculina en el hogar.
%\begin{itemize}
%  \item Índice de Desarrollo Humano (IDH), de tal forma que, por cada unidad que disminuye éste, un individuo tiene 5.7 veces más posibilidades de ser nini.
 % \item La variable sexo, era de esperarse que tenga efectos importantes dado que de origen más de 95\% de los ninis son mujeres, entonces, no es extraño que haya resultado que las mujeres tengan 4.3 veces mayores posibilidades de que adquieran esa condición.
  %\item La edad es el tercer factor en importancia debido a que incrementa las posibilidades 3.4 veces por cada año adicional, por ejemplo, al pasar de 15 a 16 años.
  %\item Ocupados en el hogar, pareciera que entre más ocupados haya en el hogar, más inhibe la posibilidad de convertirse en nini dentro de éste, pues por cada ocupado que se reste en el hogar existen casi dos veces más posibilidades de la presencia de aquéllos.
  %\item Jefatura masculina en el hogar, esta incrementa las posibilidades de presencia de ninis 1.4 veces.
%\end{itemize} 

%%Como se menciona la educación y el desempleo son dos grandes contribuyentes a la formación de un nini, debido a que los jóvenes no encuentran empleos adecuados a sus capacidades y gustos. En Parametría se reportan los resultados de una encuesta de opinión en viviendas y donde 58 por ciento de los entrevistados opina que para los ninis resulta más atractivo entrar a las filas del narcotráfico que conseguir un trabajo o asistir a la escuela. \cite{Parametria} Además, en su análisis, Escobedo \cite{JEB} menciona que 80 por ciento de los ninis ha participado en actos de violencia, aunado a esto una encuesta realizada por la firma OCCMundial ocho de cada diez jóvenes mexicanos no se inscribieron a una universidad y 42 por ciento no lo hizo porque no pudo pagar una licenciatura de estudios presenciales por otra parte seis de cada diez jóvenes abandonaron sus estudios superiores por falta de dinero. \cite{OCC} \cite{Forbes_Universidad} \\
%%
%De acuerdo con Forbes México, el gasto en educación universitaria no es prioridad para el gobierno, ya que en 2016 el gobierno de México destino 97.2 por ciento del gasto federalizado de educación al nivel básico, y menos del uno por ciento va para educación superior, según el Centro de Investigación Económica y Presupuestaria. De igual manera, el gasto en nivel básico entre las entidades no corresponde a la población educativa, pues Baja California, Chiapas, Chihuahua, Guanajuato, Jalisco, Estado de México, Nuevo León, Puebla, Sonora, Tabasco y Yucatán reciben recursos en una proporción del gasto inferior a los alumnos que tienen. Por el contrario, Aguascalientes, Baja California Sur, Campeche, Coahuila, Colima, Ciudad de México, Durango, Guerrero, Hidalgo, Michoacán, Morelos, Nayarit, Oaxaca, Querétaro, Quintana Roo, San Luis Potosí, Sinaloa, Tamaulipas, Tlaxcala, Veracruz y Zacatecas reciben recursos en una proporción del gasto educativo por aportaciones superior a la cantidad de estudiantes. \cite{Forbes_Gasto_Educacion}\\
 