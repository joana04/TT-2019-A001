%\addbibresource{referencias.bib}
\newpage
\section{Problemática}

Un joven debería ser parte de la Población Económicamente Activa (PEA), o bien, se debería preparar para entrar a ella. De no ser así, %si ese joven no tiene un empleo, ya sea formal o informal, y tampoco se está preparando en alguna institución educativa para ingresar a la PEA\\ 
se le clasifica como nini o NEET ( \textit{Youth Not in Employment, Education or Trainig}) por sus siglas en inglés. \cite{BenitoDuranRomo} \cite{OECD1}.\\

%Existen numerosos factores que contribuyen a la formación de un NINI, entre ellos se encuentran el Índice de Desarrollo Humano (IDH) del municipio de residencia del individuo, sexo, edad, número de ocupados en el hogar y, en menor medida, jefatura masculina en el hogar, la educación y el desempleo, siendo estos dos últimos los más grandes contribuyentes.\\


% agregue lo de "y una de las principales causas de éste problema es"
El desempleo se considera determínante en la condición de convertirse en NINI y una de las principales causas de éste problema es debido a que los jóvenes no encuentran empleos adecuados a sus capacidades y gustos, así mismo las empresas no los ven como posibles postulantes para sus vacantes.

A pesar de esto los jóvenes intentan encontrar alternativas para encontrar empleo o continuar con sus estudios como lo informó OCCMundial en un sondeo realizado a más de 500 jóvenes, 72 \% estaba interesado en estudios universitarios en línea o educación a distancia, y 28 \% en opciones presenciales. \cite{Forbes_Universidad}\cite{Parametria}\cite{OCC}

De acuerdo con OCCmundial en una encuesta que realizo a los ninis, el 75\% de los participantes contestaron que no son tomados en cuenta por las empresas para que se incorporen al campo laboral. Proponemos brindar una estrategia diferente para la búsqueda de empleo que asocie a los solicitantes con la oferta de oportunidades, basándose no solo en la formación profesional, sino también en las preferencias del solicitante considerando a aquellos ninis que no cuentan con un perfil altamente calificado académicamente buscando en medida de lo posible al candidato idóneo.\\

%%https://www.occ.com.mx/blog/mexicanas-denuncian-problema-equidad/

Esta nueva estrategia ayudaría a los jóvenes que no se están desarrollando ni en el área laboral ni académica a obtener nuevas oportunidades, al gobierno a tener un mayor control y conocimiento del problema y a las empresas a ampliar las opciones de posibles candidatos para cubrir las vacantes.
\\

%En Parametría (Empresa dedicada a la investigación estratégica de la opinión y análisis de resultados) se reportan los resultados de una encuesta de opinión en viviendas y donde 58 \% de los entrevistados opina que para los ninis resulta más atractivo entrar a las filas del narcotráfico que conseguir un trabajo o asistir a la escuela. \cite{Parametria}.\\
 
% Además, en su análisis, Escobedo \cite{JEB} menciona que 80 \% de los ninis ha participado en actos de violencia, aunado a esto, una encuesta realizada por la firma OCCMundial ocho de cada diez jóvenes mexicanos no se inscribieron a una universidad y 42 \% no lo hizo porque no pudo pagar una licenciatura de estudios presenciales por otra parte seis de cada diez jóvenes abandonaron sus estudios superiores por falta de dinero. \cite{OCC} \cite{Forbes_Universidad} \\




\bigskip
