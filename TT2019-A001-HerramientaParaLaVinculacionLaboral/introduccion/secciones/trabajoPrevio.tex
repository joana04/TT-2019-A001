\section{Estado del arte}

\subsection{Bono de Impacto Social (BIS) Colombia}

    Colombia, al igual que la mayoría de los países de América Latina y el Caribe, enfrenta un reto enorme para su desarrollo, se estima que la cifra de ninis de entre 15 y 24 años asciende a 582.000 en las principales ciudades del país, como Bogotá, Medellín o Cali, según datos del Departamento de Planeación Nacional de Colombia. Características como experiencia laboral, bajos niveles de educación y bajo desarrollo de habilidades importantes para integrarse al mundo laboral son constantes en este grupo de jóvenes, sin embargo, a muchos les gustaría trabajar, los hombres de este rango de edad enfrentan una tasa de desempleo de 19,2\%, más del doble que la media nacional, y las mujeres de 23,5\%, según cifras oficiales de 2017.\\
    
    El reto a enfrentar de estos programas de formación es que los jóvenes cumplan con las necesidades específicas de las empresas en sus vacantes además de tomar en cuenta que la mayoría de los jóvenes son víctimas del conflicto armado o son madres con hijos pequeños en situación de pobreza, tienen necesidades especiales o brechas de habilidades que requieren una mayor atención.\\

    BID Lab, el gobierno de Colombia, la Cooperación Suiza de SECO, las fundaciones Corona, Mario Santo Domingo y Bolívar Davivienda desarrollaron un programa pionero en América Latina y el Caribe para incrementar la participación de poblaciones vulnerables en el empleo formal. A nivel mundial, las intervenciones de empleo para poblaciones vulnerables más exitosas han logrado colocar a entre 20\%-32\% de ellas; sin embargo, el primer piloto de este proyecto logró colocar en un empleo formal a un 46\% de esta población.\\
    
    El programa funciona a través de un Bono de Impacto Social (BIS), un innovador instrumento financiero que invita a inversionistas privados a invertir para llevar a cabo proyectos sociales. El potencial de los Bonos de Impacto Social es enorme, se pueden utilizar para cubrir a otras poblaciones vulnerables en sectores como educación, salud, nutrición o embarazo adolescente. Aunque también es importante realizar diversos pilotos en escalas pequeñas para generar aprendizajes. Además, no es una solución que se puede aplicar a todos los problemas sociales: es necesario tener o poder crear una infraestructura de datos para verificar los resultados de manera precisa. \cite{BIS_Colombia} \\

    

\subsection{Programa jovenes construyendo el futuro}

    Jóvenes Construyendo el Futuro es un programa que busca que miles de jóvenes entre 18 a 29 años de edad puedan capacitarse en el trabajo. El Gobierno de México les otorgará una beca mensual de 3,600 pesos para que se capaciten durante un año. Es la oportunidad para que empresas, instituciones públicas y organizaciones sociales los capaciten para que desarrollen habilidades, aprovechen su talento y comiencen su experiencia laboral.\\
    
    Para formalizar su inscripción al Programa, los/las solicitantes deberán acudir personalmente a las oficinas designadas por la STPS (Secretaria del Tabajo y Prevención Social) o a través de la Plataforma Digital en la página: jovenesconstruyendoelfuturo.stps.gob.mx. En el trámite de Solicitud inscripción del becario, se deberán entregar copia simple legible de los siguientes documentos,en caso de duda se pedirá original para cotejo.\\
    
    \begin{itemize}
    \item  CURP. 
    \item  Identificación oficial, tal como cartilla del servicio militar nacional, cédula profesional, pasaporte, credencial para votar con fotografía. 
    \item  En caso de requerirlo comprobante de domicilio actual (no anterior a tres meses): recibo de luz, agua, predial o teléfono, o en su caso, escrito libre de la autoridad local en el que se valide la residencia del solicitante. 
    \item  En caso de requerirlo, certificado o comprobante del último grado de estudios. 
    \item  Auto-fotografía de rostro del solicitante. En caso de personas extranjeras, se deberá presentar el documento oficial que acredite su legal estancia en el país expedido por las autoridades migratorias. 
    \end{itemize} 
    
    Una vez entregada y cotejada la documentación requerida, el/la solicitante procederá al llenado de los formatos y cuestionarios necesarios para la generación de un perfil referente a sus intereses y aptitudes.
    A partir de esta información, el Programa realizará un proceso de análisis de información con el objeto de presentar al solicitante las ofertas de capacitación, en las que podrá elegir entre las opciones disponibles.\\
    
    En caso de no existir ofertas de capacitación disponibles al momento de la inscripción, se le notificará cuando exista un espacio disponible, de acuerdo al orden de prelación, siempre y cuando se cubran los requisitos y documentación señalados.
    Una vez que el solicitante elija una oferta de capacitación, se le informará de los requisitos que deberá cubrir para aplicar a la Beca, así como de los beneficios que gozará por formar parte del Programa.\\
    
    Cuando el/la solicitante elija la opción de capacitación, el Programa notificará al Centro de Trabajo seleccionado en un plazo no mayor a 10 días hábiles y le será proporcionada información general del/la becario(a) y del perfil de capacitación elegido. La formalización de ingreso al Programa se completa con la aceptación, por parte del/de la solicitante, de la carta donde se compromete a cumplir con los Lineamientos del Programa y el plan de capacitación proporcionado por el Centro de Trabajo; así como, un comprobante de inscripción que incluye la fecha y hora en la que deberá presentarse en el domicilio del Centro de Trabajo, misma que deberá presentar al acudir a dicha cita. El inicio de los programas de capacitación en los Centros de Trabajo serán los días 1 y 16 de cada mes o su equivalente al día hábil posterior.\cite{JCF}
    \\
    
   Los Representantes de los Centros de Trabajo interesados en participar en el Programa podrán realizar el registro a través de la Plataforma Digital, a través de los Servidores de la Nación o acudiendo a las oficinas designadas por la STPS\cite{JCF}.
   \\
   Los Representantes de los Centros de Trabajo deberán entregar:
   \begin{itemize}
    \item Personas Morales:
        \begin{itemize}
        \item Acta constitutiva otorgada ante fedatario público que acredite la existencia de la persona moral.
        \item Constancia de inscripción ante el RFC 
        \item Identificación oficial vigente del representante legal o apoderado del Centro de Trabajo. 
        \item Documento otorgado ante fedatario público, que acredite la personalidad del representante legal o apoderado del Centro de Trabajo. V. Comprobante de domicilio del Centro de Trabajo o del Domicilio Fiscal. 
        \item Fotografías del exterior e interior del Centro de Trabajo; es decir, del lugar donde va a realizar la capacitación, que a su juicio sirvan para acreditar la existencia del Centro de Trabajo. 
        \end{itemize} 
    \item  Personas Físicas:
        \begin{itemize}
        \item Identificación oficial vigente. 
        \item  Constancia de registro ante en el Registro Federal de Contribuyentes o de su Clave Única de Registro de Población.
        \item  Comprobante de domicilio del Centro de Trabajo.
        \item  Fotografías del exterior del Centro de Trabajo, del lugar donde va a realizar la capacitación que a su juicio sirvan para acreditar la existencia del Centro de Trabajo. 
        \end{itemize} 
        
    \item  El plan de capacitación que corresponda a cada una de las ofertas de capacitación que desee registrar, dichos planes deberán tener una duración de máximo doce meses y cumplir con las características descritas en el numeral Décimo Primero de los lineamientos del programa.
   \item Datos de contacto de la persona que fungirá como tutor(a) para cada plan de capacitación. 
    \end{itemize} 
   
    
    La inscripción al programa se formalizará con la emisión de un acuse electrónico que contiene:
    \begin{itemize}
        \item  Folio de registro. 
        \item Nombre, denominación o razón social. 
        \item Registro Federal de Contribuyentes o Clave Única de Registro de Población. 
        \item Fecha y hora de emisión. 
        \item Cadena digital. 
        \item Código QR. 
    \end{itemize}    
\bigskip
