\chapter{Lineamientos}
Los presentes lineamientos tienen por objeto regir la operación de la Herramienta para la Vicnualción Laboral en las 32 entidades federativas del país.

\section{Requisistos y documentación}

\begin{itemize}
    \item Becarios
        \begin{itemize}
            \item Requisitos 
                \begin{itemize}
                    \item Edad entre 16 y 24 años al momento del registro. 
                    \item No estar trabajando ni estudiando. 
                    \item Inscribirse en la Herramienta para la vinculación laboral, para la entrega de la información y documentación requerida en la plataforma, así como el llenado de formatos y cuestionarios necesarios para su inscripción. 
                    \item Autorizar el tratamiento de sus datos personales de acuerdo con la normatividad vigente en la materia.
                \end{itemize}
            \item Documentación
                \begin{itemize}
                    \item Acta de nacimiento. 
                    \item CURP. 
                    \item Identificación oficial, tal como cartilla del servicio militar nacional, cédula profesional, pasaporte, credencial para votar con fotografía. 
                    \item Comprobante de domicilio actual (no anterior a tres meses): recibo de luz, agua, predial o teléfono, o en su caso, escrito libre de la autoridad local en el que se valide la residencia del solicitante. 
                    \item En caso de requerirlo, certificado o comprobante del último grado de estudios. 
                \end{itemize}
        \end{itemize}
    \item Empresas (Personas morales y físicas)
        \begin{itemize}
            \item Requisitos
            \begin{itemize}
                \item Inscribirse en la Herramienta para la vinculación laboral; 
                \item Designar un tutor; 
                \item Comprobante de inscripción en el Registro Federal de Contribuyentes (opcional); 
                \item Señalar el giro del Centro de Trabajo.
            \end{itemize}
            \item Documentación
            \begin{itemize}
                \item Constancia de inscripción ante el RFC 
                \item. Identificación oficial vigente del representante legal o apoderado del Centro de Trabajo. 
                \item Comprobante de domicilio del Centro de Trabajo o del Domicilio Fiscal. 
                
            \end{itemize}
        \end{itemize}
\end{itemize}

\section{Inscripción}
Los solicitantes becarios y las empresas se registrarán por medio de la plataforma, ingresarán toda su información y documentos requeridos. Está información será verificada por la SEP, el IMSS, el SAT y por la STPS para comprobar la identidad de los solicitantes.\\
Que el usuario se registre en el sistema no quiere decir que se aceptado en el programa.\\
\section{Perfilamiento y registro de vacantes}
Una vez que el usuario es aceptado en el programa, el becario comenzara a contestar el o los cuestionarios para realizar su perfilamiento y análisis.\\
Las empresas que sean aceptadas para dar capacitaciones podrán registrar las vacantes con sus respectivos programas de capacitación y el perfil necesario.\\
\section{Asociación }

El sistema una vez que tiene los becarios y las vacantes con sus perfiles respectivos, procede a asociarlos tratando en medida de lo posible buscar el candidato idóneo para las vacantes.

\section{Contratación }

El proceso de contratación será llevado por la empresa, solamente la plataforma tendrá el registro de la contratación para darle seguimiento.

\section{Seguimiento }

Cualquier cambio en el estatus del becario, deberá ser registrado en el sistema, para llevar un control y seguimiento del programa.


