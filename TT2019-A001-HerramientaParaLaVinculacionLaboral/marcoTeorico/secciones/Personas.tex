\newpage
\section{Personas}
Una persona es un individuo de la especie humana que cuenta con una identidad\cite{Personas_1}.

Un joven que una vez finalizada la enseñanza obligatoria no se sigue formando ni tampoco tiene trabajo es considerado NINI (Ni estudia Ni trabaja) o NEET (youth Not in Employment, Education or Trainig) por us siglas en ingles\cite{BenitoDuranRomo} \cite{OECD1}.

\bigskip
\section{Datos personales}
    Los datos personales son toda aquella información que se relaciona con nuestra persona y que nos identifica o nos hace identificables\cite{Personas_DP}. Nos dan identidad, nos describen y precisan:
    \begin{itemize}
      \item Edad
      \item Domicilio
      \item Número telefónico
      \item Correo electrónico personal
      \item Trayectoria académica, laboral o profesional
      \item Número de seguridad social
      \item CURP, entre otros.
    \end{itemize}
También describen aspectos más sensibles o delicados, como es el caso de:
    \begin{itemize}
      \item Forma de pensar
      \item Estado de salud
      \item Origen étnico y racial
      \item Características físicas (ADN, huella digital)
      \item Creencias o convicciones religiosas o filosóficas
      \item Ideología y opiniones políticas
      \item Preferencias sexuales, entre otros.
    \end{itemize}
    
    Los datos personales siempre son tuyos, pero a veces es necesario que los proporciones a otros para hacer un trámite, comprar un producto o contratar un servicio. De manera común, tanto particulares (médicos, bancos, hoteles, empresas de telefonía móvil, aseguradoras, etc.) como Sujetos Obligados (oficinas de tránsito, catastro, escuelas y hospitales públicos, tribunales, procuradurías, entre otros) recaban nuestros datos.

\subsection{Normatividad}

    Ley Federal de Protección de Datos Personales en Posesión de los Particulares (LFPDPPP) aprobada por la Asamblea Legislativa el 27 de abril del 2010 y publicada en el Diario Oficial de la Federación el 5 de julio del 2010.\\
    
    La LFPDPPP es de orden público y de observancia general en toda la República y
    tiene por objeto la protección de los datos personales en posesión de los particulares, con la finalidad de regular su tratamiento legítimo, controlado e informado, a efecto de garantizar la privacidad y el derecho a
    la autodeterminación informativa de las personas\cite{Personas_Normatividad}.\\\\
    La LFPDPPP establece:
     \begin{itemize}
      \item Capítulo I Generalidades
      \item Capítulo II De los Principios de Protección de Datos Personales
      \item Capítulo III De los Derechos de los Titulares de Datos Personales
      \item Capítulo IV Del Ejercicio de los Derechos de Acceso, Rectificación, Cancelación y Oposición
      \item Capítulo V De la Transferencia de Datos
      \item Capítulo VI De las Autoridades
      \item Capítulo VII Del Procedimiento de Protección de Derechos
      \item Capítulo VIII Del Procedimiento de Verificación
      \item Capítulo IX Del Procedimiento de Imposición de Sanciones
      \item Capítulo X De las Infracciones y Sanciones
      \item Capítulo XI De los Delitos en Materia del Tratamiento Indebido de Datos Personales
    \end{itemize}
    
\subsection{Categorías de datos personales}
\begin{center}
\begin{longtable}{|p{4cm}|p{10cm}|} \hline
   %\begin{tabular}[c]{|p{4cm}|p{10cm}|} \hline
     {Categoría} & {Tipo de datos} \\ \hline \hline
     Datos identificativos & El nombre, domicilio, teléfono particular, teléfono celular, firma, clave del Registro Federal de Contribuyentes (RFC), Clave Única de Registro de Población (CURP), Clave de elector, Matrícula del Servicio Militar Nacional, número de pasaporte, fecha de nacimiento y demás análogos.  \\ \hline
     
     Datos electrónicos  & Las direcciones electrónicas, el correo electrónico, dirección IP , dirección MAC,  nombre del usuario, contraseñas, firma electrónica; o cualquier otra información empleada por la persona, para su identificación en Internet.  \\  \hline
     
     Datos laborales & Documentos de reclutamiento y selección, capacitación, referencias laborales y personales, solicitud de empleo, hoja de servicio y demás análogos.\\  \hline
     
     Datos académicos & Trayectoria educativa, calificaciones, títulos, cédula profesional, certificados y reconocimientos y demás análogos.\\  \hline
     
     Datos de salud & El expediente clínico de cualquier atención médica, referencias o descripción de sintomatologías, detección de enfermedades, discapacidades, intervenciones quirúrgicas, vacunas,  así como el estado físico o mental de la persona. \\  \hline
     
     Datos patrimoniales & Los correspondientes a bienes muebles e inmuebles, información fiscal, historial crediticio, ingresos y egresos, cuentas bancarias, seguros, fianzas, referencias personales y demás análogos. \\  \hline
     Datos sobre procedimiento administrativo &  La información relativa a una persona que se encuentre sujeta a un procedimiento administrativo seguido en forma de juicio o jurisdiccional en materia laboral, civil, penal, fiscal, administrativa o de cualquier otra rama del Derecho. \\  \hline
     
     Datos de tránsito y movimientos migratorios & Información relativa al tránsito de las personas dentro y fuera del país, así como información migratoria.\\  \hline
     
    Datos biométricos  & Huellas dactilares, ADN, geometría de la mano, características de iris y retina y demás análogos.\\  \hline
    
    Datos sensibles & Origen étnico o racial, características morales o emocionales, ideología y opiniones políticas, creencias, convicciones religiosas, filosóficas, la pertenencia a sindicatos, la salud y preferencia sexual \\  \hline
    Datos de naturaleza pública & Aquellos que por mandato legal sean accesibles al público \\  \hline
 %  \end{tabular} 
   \caption{Categorías de datos personales}
 \end{longtable}
 \end{center}




%%\subsection{} 
%%\newpage
\section{Personas}
Una persona es un individuo de la especie humana que cuenta con una identidad\cite{Personas_1}.

Un joven que una vez finalizada la enseñanza obligatoria no se sigue formando ni tampoco tiene trabajo es considerado NINI (Ni estudia Ni trabaja) o NEET (youth Not in Employment, Education or Trainig) por us siglas en ingles\cite{BenitoDuranRomo} \cite{OECD1}.

\bigskip
\section{Datos personales}
    Los datos personales son toda aquella información que se relaciona con nuestra persona y que nos identifica o nos hace identificables\cite{Personas_DP}. Nos dan identidad, nos describen y precisan:
    \begin{itemize}
      \item Edad
      \item Domicilio
      \item Número telefónico
      \item Correo electrónico personal
      \item Trayectoria académica, laboral o profesional
      \item Número de seguridad social
      \item CURP, entre otros.
    \end{itemize}
También describen aspectos más sensibles o delicados, como es el caso de:
    \begin{itemize}
      \item Forma de pensar
      \item Estado de salud
      \item Origen étnico y racial
      \item Características físicas (ADN, huella digital)
      \item Creencias o convicciones religiosas o filosóficas
      \item Ideología y opiniones políticas
      \item Preferencias sexuales, entre otros.
    \end{itemize}
    
    Los datos personales siempre son tuyos, pero a veces es necesario que los proporciones a otros para hacer un trámite, comprar un producto o contratar un servicio. De manera común, tanto particulares (médicos, bancos, hoteles, empresas de telefonía móvil, aseguradoras, etc.) como Sujetos Obligados (oficinas de tránsito, catastro, escuelas y hospitales públicos, tribunales, procuradurías, entre otros) recaban nuestros datos.

\subsection{Normatividad}

    Ley Federal de Protección de Datos Personales en Posesión de los Particulares (LFPDPPP) aprobada por la Asamblea Legislativa el 27 de abril del 2010 y publicada en el Diario Oficial de la Federación el 5 de julio del 2010.\\
    
    La LFPDPPP es de orden público y de observancia general en toda la República y
    tiene por objeto la protección de los datos personales en posesión de los particulares, con la finalidad de regular su tratamiento legítimo, controlado e informado, a efecto de garantizar la privacidad y el derecho a
    la autodeterminación informativa de las personas\cite{Personas_Normatividad}.\\\\
    La LFPDPPP establece:
     \begin{itemize}
      \item Capítulo I Generalidades
      \item Capítulo II De los Principios de Protección de Datos Personales
      \item Capítulo III De los Derechos de los Titulares de Datos Personales
      \item Capítulo IV Del Ejercicio de los Derechos de Acceso, Rectificación, Cancelación y Oposición
      \item Capítulo V De la Transferencia de Datos
      \item Capítulo VI De las Autoridades
      \item Capítulo VII Del Procedimiento de Protección de Derechos
      \item Capítulo VIII Del Procedimiento de Verificación
      \item Capítulo IX Del Procedimiento de Imposición de Sanciones
      \item Capítulo X De las Infracciones y Sanciones
      \item Capítulo XI De los Delitos en Materia del Tratamiento Indebido de Datos Personales
    \end{itemize}
    
\subsection{Categorías de datos personales}
\begin{center}
\begin{longtable}{|p{4cm}|p{10cm}|} \hline
   %\begin{tabular}[c]{|p{4cm}|p{10cm}|} \hline
     {Categoría} & {Tipo de datos} \\ \hline \hline
     Datos identificativos & El nombre, domicilio, teléfono particular, teléfono celular, firma, clave del Registro Federal de Contribuyentes (RFC), Clave Única de Registro de Población (CURP), Clave de elector, Matrícula del Servicio Militar Nacional, número de pasaporte, fecha de nacimiento y demás análogos.  \\ \hline
     
     Datos electrónicos  & Las direcciones electrónicas, el correo electrónico, dirección IP , dirección MAC,  nombre del usuario, contraseñas, firma electrónica; o cualquier otra información empleada por la persona, para su identificación en Internet.  \\  \hline
     
     Datos laborales & Documentos de reclutamiento y selección, capacitación, referencias laborales y personales, solicitud de empleo, hoja de servicio y demás análogos.\\  \hline
     
     Datos académicos & Trayectoria educativa, calificaciones, títulos, cédula profesional, certificados y reconocimientos y demás análogos.\\  \hline
     
     Datos de salud & El expediente clínico de cualquier atención médica, referencias o descripción de sintomatologías, detección de enfermedades, discapacidades, intervenciones quirúrgicas, vacunas,  así como el estado físico o mental de la persona. \\  \hline
     
     Datos patrimoniales & Los correspondientes a bienes muebles e inmuebles, información fiscal, historial crediticio, ingresos y egresos, cuentas bancarias, seguros, fianzas, referencias personales y demás análogos. \\  \hline
     Datos sobre procedimiento administrativo &  La información relativa a una persona que se encuentre sujeta a un procedimiento administrativo seguido en forma de juicio o jurisdiccional en materia laboral, civil, penal, fiscal, administrativa o de cualquier otra rama del Derecho. \\  \hline
     
     Datos de tránsito y movimientos migratorios & Información relativa al tránsito de las personas dentro y fuera del país, así como información migratoria.\\  \hline
     
    Datos biométricos  & Huellas dactilares, ADN, geometría de la mano, características de iris y retina y demás análogos.\\  \hline
    
    Datos sensibles & Origen étnico o racial, características morales o emocionales, ideología y opiniones políticas, creencias, convicciones religiosas, filosóficas, la pertenencia a sindicatos, la salud y preferencia sexual \\  \hline
    Datos de naturaleza pública & Aquellos que por mandato legal sean accesibles al público \\  \hline
 %  \end{tabular} 
   \caption{Categorías de datos personales}
 \end{longtable}
 \end{center}




%%\subsection{} 
%%\newpage
\section{Personas}
Una persona es un individuo de la especie humana que cuenta con una identidad\cite{Personas_1}.

Un joven que una vez finalizada la enseñanza obligatoria no se sigue formando ni tampoco tiene trabajo es considerado NINI (Ni estudia Ni trabaja) o NEET (youth Not in Employment, Education or Trainig) por us siglas en ingles\cite{BenitoDuranRomo} \cite{OECD1}.

\bigskip
\section{Datos personales}
    Los datos personales son toda aquella información que se relaciona con nuestra persona y que nos identifica o nos hace identificables\cite{Personas_DP}. Nos dan identidad, nos describen y precisan:
    \begin{itemize}
      \item Edad
      \item Domicilio
      \item Número telefónico
      \item Correo electrónico personal
      \item Trayectoria académica, laboral o profesional
      \item Número de seguridad social
      \item CURP, entre otros.
    \end{itemize}
También describen aspectos más sensibles o delicados, como es el caso de:
    \begin{itemize}
      \item Forma de pensar
      \item Estado de salud
      \item Origen étnico y racial
      \item Características físicas (ADN, huella digital)
      \item Creencias o convicciones religiosas o filosóficas
      \item Ideología y opiniones políticas
      \item Preferencias sexuales, entre otros.
    \end{itemize}
    
    Los datos personales siempre son tuyos, pero a veces es necesario que los proporciones a otros para hacer un trámite, comprar un producto o contratar un servicio. De manera común, tanto particulares (médicos, bancos, hoteles, empresas de telefonía móvil, aseguradoras, etc.) como Sujetos Obligados (oficinas de tránsito, catastro, escuelas y hospitales públicos, tribunales, procuradurías, entre otros) recaban nuestros datos.

\subsection{Normatividad}

    Ley Federal de Protección de Datos Personales en Posesión de los Particulares (LFPDPPP) aprobada por la Asamblea Legislativa el 27 de abril del 2010 y publicada en el Diario Oficial de la Federación el 5 de julio del 2010.\\
    
    La LFPDPPP es de orden público y de observancia general en toda la República y
    tiene por objeto la protección de los datos personales en posesión de los particulares, con la finalidad de regular su tratamiento legítimo, controlado e informado, a efecto de garantizar la privacidad y el derecho a
    la autodeterminación informativa de las personas\cite{Personas_Normatividad}.\\\\
    La LFPDPPP establece:
     \begin{itemize}
      \item Capítulo I Generalidades
      \item Capítulo II De los Principios de Protección de Datos Personales
      \item Capítulo III De los Derechos de los Titulares de Datos Personales
      \item Capítulo IV Del Ejercicio de los Derechos de Acceso, Rectificación, Cancelación y Oposición
      \item Capítulo V De la Transferencia de Datos
      \item Capítulo VI De las Autoridades
      \item Capítulo VII Del Procedimiento de Protección de Derechos
      \item Capítulo VIII Del Procedimiento de Verificación
      \item Capítulo IX Del Procedimiento de Imposición de Sanciones
      \item Capítulo X De las Infracciones y Sanciones
      \item Capítulo XI De los Delitos en Materia del Tratamiento Indebido de Datos Personales
    \end{itemize}
    
\subsection{Categorías de datos personales}
\begin{center}
\begin{longtable}{|p{4cm}|p{10cm}|} \hline
   %\begin{tabular}[c]{|p{4cm}|p{10cm}|} \hline
     {Categoría} & {Tipo de datos} \\ \hline \hline
     Datos identificativos & El nombre, domicilio, teléfono particular, teléfono celular, firma, clave del Registro Federal de Contribuyentes (RFC), Clave Única de Registro de Población (CURP), Clave de elector, Matrícula del Servicio Militar Nacional, número de pasaporte, fecha de nacimiento y demás análogos.  \\ \hline
     
     Datos electrónicos  & Las direcciones electrónicas, el correo electrónico, dirección IP , dirección MAC,  nombre del usuario, contraseñas, firma electrónica; o cualquier otra información empleada por la persona, para su identificación en Internet.  \\  \hline
     
     Datos laborales & Documentos de reclutamiento y selección, capacitación, referencias laborales y personales, solicitud de empleo, hoja de servicio y demás análogos.\\  \hline
     
     Datos académicos & Trayectoria educativa, calificaciones, títulos, cédula profesional, certificados y reconocimientos y demás análogos.\\  \hline
     
     Datos de salud & El expediente clínico de cualquier atención médica, referencias o descripción de sintomatologías, detección de enfermedades, discapacidades, intervenciones quirúrgicas, vacunas,  así como el estado físico o mental de la persona. \\  \hline
     
     Datos patrimoniales & Los correspondientes a bienes muebles e inmuebles, información fiscal, historial crediticio, ingresos y egresos, cuentas bancarias, seguros, fianzas, referencias personales y demás análogos. \\  \hline
     Datos sobre procedimiento administrativo &  La información relativa a una persona que se encuentre sujeta a un procedimiento administrativo seguido en forma de juicio o jurisdiccional en materia laboral, civil, penal, fiscal, administrativa o de cualquier otra rama del Derecho. \\  \hline
     
     Datos de tránsito y movimientos migratorios & Información relativa al tránsito de las personas dentro y fuera del país, así como información migratoria.\\  \hline
     
    Datos biométricos  & Huellas dactilares, ADN, geometría de la mano, características de iris y retina y demás análogos.\\  \hline
    
    Datos sensibles & Origen étnico o racial, características morales o emocionales, ideología y opiniones políticas, creencias, convicciones religiosas, filosóficas, la pertenencia a sindicatos, la salud y preferencia sexual \\  \hline
    Datos de naturaleza pública & Aquellos que por mandato legal sean accesibles al público \\  \hline
 %  \end{tabular} 
   \caption{Categorías de datos personales}
 \end{longtable}
 \end{center}




%%\subsection{} 
%%\newpage
\section{Personas}
Una persona es un individuo de la especie humana que cuenta con una identidad\cite{Personas_1}.

Un joven que una vez finalizada la enseñanza obligatoria no se sigue formando ni tampoco tiene trabajo es considerado NINI (Ni estudia Ni trabaja) o NEET (youth Not in Employment, Education or Trainig) por us siglas en ingles\cite{BenitoDuranRomo} \cite{OECD1}.

\bigskip
\section{Datos personales}
    Los datos personales son toda aquella información que se relaciona con nuestra persona y que nos identifica o nos hace identificables\cite{Personas_DP}. Nos dan identidad, nos describen y precisan:
    \begin{itemize}
      \item Edad
      \item Domicilio
      \item Número telefónico
      \item Correo electrónico personal
      \item Trayectoria académica, laboral o profesional
      \item Número de seguridad social
      \item CURP, entre otros.
    \end{itemize}
También describen aspectos más sensibles o delicados, como es el caso de:
    \begin{itemize}
      \item Forma de pensar
      \item Estado de salud
      \item Origen étnico y racial
      \item Características físicas (ADN, huella digital)
      \item Creencias o convicciones religiosas o filosóficas
      \item Ideología y opiniones políticas
      \item Preferencias sexuales, entre otros.
    \end{itemize}
    
    Los datos personales siempre son tuyos, pero a veces es necesario que los proporciones a otros para hacer un trámite, comprar un producto o contratar un servicio. De manera común, tanto particulares (médicos, bancos, hoteles, empresas de telefonía móvil, aseguradoras, etc.) como Sujetos Obligados (oficinas de tránsito, catastro, escuelas y hospitales públicos, tribunales, procuradurías, entre otros) recaban nuestros datos.

\subsection{Normatividad}

    Ley Federal de Protección de Datos Personales en Posesión de los Particulares (LFPDPPP) aprobada por la Asamblea Legislativa el 27 de abril del 2010 y publicada en el Diario Oficial de la Federación el 5 de julio del 2010.\\
    
    La LFPDPPP es de orden público y de observancia general en toda la República y
    tiene por objeto la protección de los datos personales en posesión de los particulares, con la finalidad de regular su tratamiento legítimo, controlado e informado, a efecto de garantizar la privacidad y el derecho a
    la autodeterminación informativa de las personas\cite{Personas_Normatividad}.\\\\
    La LFPDPPP establece:
     \begin{itemize}
      \item Capítulo I Generalidades
      \item Capítulo II De los Principios de Protección de Datos Personales
      \item Capítulo III De los Derechos de los Titulares de Datos Personales
      \item Capítulo IV Del Ejercicio de los Derechos de Acceso, Rectificación, Cancelación y Oposición
      \item Capítulo V De la Transferencia de Datos
      \item Capítulo VI De las Autoridades
      \item Capítulo VII Del Procedimiento de Protección de Derechos
      \item Capítulo VIII Del Procedimiento de Verificación
      \item Capítulo IX Del Procedimiento de Imposición de Sanciones
      \item Capítulo X De las Infracciones y Sanciones
      \item Capítulo XI De los Delitos en Materia del Tratamiento Indebido de Datos Personales
    \end{itemize}
    
\subsection{Categorías de datos personales}
\begin{center}
\begin{longtable}{|p{4cm}|p{10cm}|} \hline
   %\begin{tabular}[c]{|p{4cm}|p{10cm}|} \hline
     {Categoría} & {Tipo de datos} \\ \hline \hline
     Datos identificativos & El nombre, domicilio, teléfono particular, teléfono celular, firma, clave del Registro Federal de Contribuyentes (RFC), Clave Única de Registro de Población (CURP), Clave de elector, Matrícula del Servicio Militar Nacional, número de pasaporte, fecha de nacimiento y demás análogos.  \\ \hline
     
     Datos electrónicos  & Las direcciones electrónicas, el correo electrónico, dirección IP , dirección MAC,  nombre del usuario, contraseñas, firma electrónica; o cualquier otra información empleada por la persona, para su identificación en Internet.  \\  \hline
     
     Datos laborales & Documentos de reclutamiento y selección, capacitación, referencias laborales y personales, solicitud de empleo, hoja de servicio y demás análogos.\\  \hline
     
     Datos académicos & Trayectoria educativa, calificaciones, títulos, cédula profesional, certificados y reconocimientos y demás análogos.\\  \hline
     
     Datos de salud & El expediente clínico de cualquier atención médica, referencias o descripción de sintomatologías, detección de enfermedades, discapacidades, intervenciones quirúrgicas, vacunas,  así como el estado físico o mental de la persona. \\  \hline
     
     Datos patrimoniales & Los correspondientes a bienes muebles e inmuebles, información fiscal, historial crediticio, ingresos y egresos, cuentas bancarias, seguros, fianzas, referencias personales y demás análogos. \\  \hline
     Datos sobre procedimiento administrativo &  La información relativa a una persona que se encuentre sujeta a un procedimiento administrativo seguido en forma de juicio o jurisdiccional en materia laboral, civil, penal, fiscal, administrativa o de cualquier otra rama del Derecho. \\  \hline
     
     Datos de tránsito y movimientos migratorios & Información relativa al tránsito de las personas dentro y fuera del país, así como información migratoria.\\  \hline
     
    Datos biométricos  & Huellas dactilares, ADN, geometría de la mano, características de iris y retina y demás análogos.\\  \hline
    
    Datos sensibles & Origen étnico o racial, características morales o emocionales, ideología y opiniones políticas, creencias, convicciones religiosas, filosóficas, la pertenencia a sindicatos, la salud y preferencia sexual \\  \hline
    Datos de naturaleza pública & Aquellos que por mandato legal sean accesibles al público \\  \hline
 %  \end{tabular} 
   \caption{Categorías de datos personales}
 \end{longtable}
 \end{center}




%%\subsection{} 
%%\input{./marcoTeorico/secciones/Personas }