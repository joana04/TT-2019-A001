\section{Sistema De Información}

Según \cite{ITSOM_SI} Información se define un conjunto de datos relacionados entre sí, es decir, una estructura de datos. Sin embargo, no cualquier estructura de datos constituye información: la información está siempre referida a objetos concretos o ideales, pero siempre bien especificados y contextualizados. Por otra parte, Sistema \cite{CAM_SYSTEM} se define como un conjunto de elementos que trabajan en conjunto con un propósito particular.\\

Entonces, un sistema de información se puede definir técnicamente como un conjunto de componentes relacionados que recolectan (o recuperan), procesan, almacenan y distribuyen información para apoyar la toma de decisiones y el control en una organización. \cite{ITSOM_SI}

\subsection{Funcionamiento de un sistema de información}
Para un sistema de información existen actividades fundamentales que producen la información que esas organizaciones necesitan para tomar decisiones, controlar operaciones, analizar problemas y crear nuevos productos o servicios. Estas actividades son:
\begin{itemize}
  \item  Entrada: captura o recolecta datos en bruto tanto del interior de la organización como de su entorno externo.
  \item Procesamiento: convierte esa entrada de datos en una forma más significativa.
  \item  Salida: transfiere la información procesada a la gente que la usará o a las actividades para las que se utilizará.
  \item Retroalimentación: la salida que se devuelve al personal adecuado de la organización para ayudarle a evaluar o corregir la etapa de entrada.
\end{itemize} %%\cite{ITSOM_SI} %% Agregar imagen parecida a https://tesis.ipn.mx/bitstream/handle/123456789/22347/Metodologia%20para%20el%20desarrollo%20de%20pruebas%20de%20software%20basada%20en%20metricas%20de%20funcionalidad%20y%20rendimiento.pdf?sequence=9&isAllowed=y pagina 14

