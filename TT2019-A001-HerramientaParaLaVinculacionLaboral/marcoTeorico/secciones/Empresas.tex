 \newpage

\section{Personas físicas}
Es un individuo que realiza cualquier actividad económica (vendedor, comerciante, empleado, etc.),  tiene obligaciones que cumplir y derechos.
Los régimenes en se que clasifican las Personas Físicas de acuerdo a sus actividades e ingresos son:

\begin{itemize}
  \item  Salarios y en general por la prestación de un servicio personal subordinado
  \item  Actividades Empresariales y Profesionales
  \item  Régimen de Incorporación Fiscal
  \item  Arrendamiento y en general por el uso o goce temporal de bienes inmuebles
  \item Enajenación de Bienes
  \item Adquisición de Bienes
  \item Intereses
  \item Obtención de Premios
  \item Dividendos y en general por las ganancias distribuidas por Personas Morales
\end{itemize}

Los régimenes fiscales \cite{SAT-FISCALES}  que una persona física puede elegir de acuerdo con las actividades empresariales que llevará a cabo son:

\begin{itemize}
  \item Régimen de Incorporación Fiscal: Pueden inscribirse aquellas personas físicas que realicen una actividad comercial o presten algún servicio por los que no requieran título profesional, siempre que sus ingresos anuales no excedan los dos millones de pesos.
  \item Actividad empresarial: pueden tributar aquellas personas físicas que obtienen ingresos por actividades comerciales (restaurantes, cafeterías, escuelas, farmacias, etc.), industriales (minería, textil y calzado, farmacéutica, construcción, etc.).
  \item Actividades Agrícolas, Ganaderas, Silvícolas y Pesqueras (sector primario): Pagarán sus impuestos en este régimen las personas físicas y morales, siempre que sus ingresos por dichas actividades representen cuando menos 90\% de sus ingresos totales.
\end{itemize} %https://www.sat.gob.mx/consulta/09788/emprendedor-conoce-los-regimenes-fiscales


\section{Personas morales}
Una persona moral es una agrupación de individuos que se unen con un fin específico, por ejemplo, una sociedad mercantil o una asociación civil.

Los régimenes fiscales  \cite{SANTANDER-PERSONAS} que una persona moral puede elegir de acuerdo con las actividads empresariales que llevara a cabo son:
\begin{itemize}
  \item Personas morales del régimen general: Se trata de sociedades mercantiles, asociaciones civiles, sociedades cooperativas de producción, instituciones de crédito, se seguros o fianzas; almacenes generales, arrendadoras financieras, uniones de crédito, sociedades de inversión de capitales, organismos descentralizados o fideicomisos con actividades empresariales, entre otras, siempre que lleven a cabo actividades lucrativas.
  \item Personas morales con fines no lucrativos: Se refiere a aquellas que no persiguen obtener una ganancia económica, por ejemplo: sociedades de inversión, administradoras de fondos para el retiro, sindicatos, cámaras de comercio e industria, colegios de profesionales, instituciones de asistencia o beneficencia y asociaciones civiles sin fines de lucro.
\end{itemize}
%https://www.santanderpyme.com.mx/detalle-noticia/personas-fisicas-o-morales-sabes-a-cual-perteneces.html
  \bigskip
\section{Obligaciones de los contribuyentes} 

Las personas morales y personas físicas que deban presentar declaraciones periódicas o que estén obligadas a expedir facturas electrónicas por los actos o actividades que realicen o por los ingresos que perciban, o que hayan abierto una cuenta a su nombre en las entidades del sistema financiero o en las sociedades cooperativas de ahorro y préstamo, en las que reciban depósitos o realicen operaciones susceptibles de ser sujetas de contribuciones, deben solicitar su inscripción en el Registro Federal de Contribuyentes, proporcionar la información relacionada con su identidad, su domicilio y, en general, sobre su situación fiscal. Asimismo, están obligadas a manifestar al Registro Federal de Contribuyentes su domicilio fiscal. Las personas morales y las personas físicas que deban presentar declaraciones periódicas o que estén obligadas a expedir facturas por los actos o actividades que realicen o por los ingresos que perciban, deben solicitar su firma electrónica. \cite{SAT-OBLIGACIONES}% http://omawww.sat.gob.mx/DerechosyObligaciones/Paginas/obligaciones_contribuyentes.htm
 \bigskip
\section{Programas de capacitación (STPS)}
Un programa de capacitación se define como la descripción detallada de un conjunto de actividades de instrucción-aprendizaje estructuradas de tal forma que conduzcan a alacanzar una serie de objetivos previamente determinados \cite{STPS-PC}.
\\
Las funciones que tiene un programa de capacitación son:
\begin{itemize}
  \item Orientar las actividades de capacitación al señalar los objetivos, actividades, técnicas y recursos que se aplicaran durante el proceso instrucción-aprendizaje.
  \item Seleccionar los contenidos al tener como parámetro el análisis de actividades de manera organizada y sistemática con base en el diagnóstico de necesidades
  \item Ofrecer al instructor la visión de conjunto del evento, permitiéndole conocer la estructura del mismo y auxiliado en la elaboración del plan de sesión
  \item Brindar al capacitando la visión total respecto a cómo será el proceso instrucción-aprendizaje durante el periodo establecido
  \item Proporcionar las bases para efectuar la evaluación del programa, es decir, la forma en que está estructurado respecto a la selección y organización de contenidos y su ubicación en relación al plan de capacitación del cual forma parte 
\end{itemize}
%https://www.gob.mx/cms/uploads/attachment/file/160973/Elaboracion_de_programas_de_capacitaci_n_Anexo_1_250_1.pdf
%%\subsection{} 
%% \newpage

\section{Personas físicas}
Es un individuo que realiza cualquier actividad económica (vendedor, comerciante, empleado, etc.),  tiene obligaciones que cumplir y derechos.
Los régimenes en se que clasifican las Personas Físicas de acuerdo a sus actividades e ingresos son:

\begin{itemize}
  \item  Salarios y en general por la prestación de un servicio personal subordinado
  \item  Actividades Empresariales y Profesionales
  \item  Régimen de Incorporación Fiscal
  \item  Arrendamiento y en general por el uso o goce temporal de bienes inmuebles
  \item Enajenación de Bienes
  \item Adquisición de Bienes
  \item Intereses
  \item Obtención de Premios
  \item Dividendos y en general por las ganancias distribuidas por Personas Morales
\end{itemize}

Los régimenes fiscales \cite{SAT-FISCALES}  que una persona física puede elegir de acuerdo con las actividades empresariales que llevará a cabo son:

\begin{itemize}
  \item Régimen de Incorporación Fiscal: Pueden inscribirse aquellas personas físicas que realicen una actividad comercial o presten algún servicio por los que no requieran título profesional, siempre que sus ingresos anuales no excedan los dos millones de pesos.
  \item Actividad empresarial: pueden tributar aquellas personas físicas que obtienen ingresos por actividades comerciales (restaurantes, cafeterías, escuelas, farmacias, etc.), industriales (minería, textil y calzado, farmacéutica, construcción, etc.).
  \item Actividades Agrícolas, Ganaderas, Silvícolas y Pesqueras (sector primario): Pagarán sus impuestos en este régimen las personas físicas y morales, siempre que sus ingresos por dichas actividades representen cuando menos 90\% de sus ingresos totales.
\end{itemize} %https://www.sat.gob.mx/consulta/09788/emprendedor-conoce-los-regimenes-fiscales


\section{Personas morales}
Una persona moral es una agrupación de individuos que se unen con un fin específico, por ejemplo, una sociedad mercantil o una asociación civil.

Los régimenes fiscales  \cite{SANTANDER-PERSONAS} que una persona moral puede elegir de acuerdo con las actividads empresariales que llevara a cabo son:
\begin{itemize}
  \item Personas morales del régimen general: Se trata de sociedades mercantiles, asociaciones civiles, sociedades cooperativas de producción, instituciones de crédito, se seguros o fianzas; almacenes generales, arrendadoras financieras, uniones de crédito, sociedades de inversión de capitales, organismos descentralizados o fideicomisos con actividades empresariales, entre otras, siempre que lleven a cabo actividades lucrativas.
  \item Personas morales con fines no lucrativos: Se refiere a aquellas que no persiguen obtener una ganancia económica, por ejemplo: sociedades de inversión, administradoras de fondos para el retiro, sindicatos, cámaras de comercio e industria, colegios de profesionales, instituciones de asistencia o beneficencia y asociaciones civiles sin fines de lucro.
\end{itemize}
%https://www.santanderpyme.com.mx/detalle-noticia/personas-fisicas-o-morales-sabes-a-cual-perteneces.html
  \bigskip
\section{Obligaciones de los contribuyentes} 

Las personas morales y personas físicas que deban presentar declaraciones periódicas o que estén obligadas a expedir facturas electrónicas por los actos o actividades que realicen o por los ingresos que perciban, o que hayan abierto una cuenta a su nombre en las entidades del sistema financiero o en las sociedades cooperativas de ahorro y préstamo, en las que reciban depósitos o realicen operaciones susceptibles de ser sujetas de contribuciones, deben solicitar su inscripción en el Registro Federal de Contribuyentes, proporcionar la información relacionada con su identidad, su domicilio y, en general, sobre su situación fiscal. Asimismo, están obligadas a manifestar al Registro Federal de Contribuyentes su domicilio fiscal. Las personas morales y las personas físicas que deban presentar declaraciones periódicas o que estén obligadas a expedir facturas por los actos o actividades que realicen o por los ingresos que perciban, deben solicitar su firma electrónica. \cite{SAT-OBLIGACIONES}% http://omawww.sat.gob.mx/DerechosyObligaciones/Paginas/obligaciones_contribuyentes.htm
 \bigskip
\section{Programas de capacitación (STPS)}
Un programa de capacitación se define como la descripción detallada de un conjunto de actividades de instrucción-aprendizaje estructuradas de tal forma que conduzcan a alacanzar una serie de objetivos previamente determinados \cite{STPS-PC}.
\\
Las funciones que tiene un programa de capacitación son:
\begin{itemize}
  \item Orientar las actividades de capacitación al señalar los objetivos, actividades, técnicas y recursos que se aplicaran durante el proceso instrucción-aprendizaje.
  \item Seleccionar los contenidos al tener como parámetro el análisis de actividades de manera organizada y sistemática con base en el diagnóstico de necesidades
  \item Ofrecer al instructor la visión de conjunto del evento, permitiéndole conocer la estructura del mismo y auxiliado en la elaboración del plan de sesión
  \item Brindar al capacitando la visión total respecto a cómo será el proceso instrucción-aprendizaje durante el periodo establecido
  \item Proporcionar las bases para efectuar la evaluación del programa, es decir, la forma en que está estructurado respecto a la selección y organización de contenidos y su ubicación en relación al plan de capacitación del cual forma parte 
\end{itemize}
%https://www.gob.mx/cms/uploads/attachment/file/160973/Elaboracion_de_programas_de_capacitaci_n_Anexo_1_250_1.pdf
%%\subsection{} 
%% \newpage

\section{Personas físicas}
Es un individuo que realiza cualquier actividad económica (vendedor, comerciante, empleado, etc.),  tiene obligaciones que cumplir y derechos.
Los régimenes en se que clasifican las Personas Físicas de acuerdo a sus actividades e ingresos son:

\begin{itemize}
  \item  Salarios y en general por la prestación de un servicio personal subordinado
  \item  Actividades Empresariales y Profesionales
  \item  Régimen de Incorporación Fiscal
  \item  Arrendamiento y en general por el uso o goce temporal de bienes inmuebles
  \item Enajenación de Bienes
  \item Adquisición de Bienes
  \item Intereses
  \item Obtención de Premios
  \item Dividendos y en general por las ganancias distribuidas por Personas Morales
\end{itemize}

Los régimenes fiscales \cite{SAT-FISCALES}  que una persona física puede elegir de acuerdo con las actividades empresariales que llevará a cabo son:

\begin{itemize}
  \item Régimen de Incorporación Fiscal: Pueden inscribirse aquellas personas físicas que realicen una actividad comercial o presten algún servicio por los que no requieran título profesional, siempre que sus ingresos anuales no excedan los dos millones de pesos.
  \item Actividad empresarial: pueden tributar aquellas personas físicas que obtienen ingresos por actividades comerciales (restaurantes, cafeterías, escuelas, farmacias, etc.), industriales (minería, textil y calzado, farmacéutica, construcción, etc.).
  \item Actividades Agrícolas, Ganaderas, Silvícolas y Pesqueras (sector primario): Pagarán sus impuestos en este régimen las personas físicas y morales, siempre que sus ingresos por dichas actividades representen cuando menos 90\% de sus ingresos totales.
\end{itemize} %https://www.sat.gob.mx/consulta/09788/emprendedor-conoce-los-regimenes-fiscales


\section{Personas morales}
Una persona moral es una agrupación de individuos que se unen con un fin específico, por ejemplo, una sociedad mercantil o una asociación civil.

Los régimenes fiscales  \cite{SANTANDER-PERSONAS} que una persona moral puede elegir de acuerdo con las actividads empresariales que llevara a cabo son:
\begin{itemize}
  \item Personas morales del régimen general: Se trata de sociedades mercantiles, asociaciones civiles, sociedades cooperativas de producción, instituciones de crédito, se seguros o fianzas; almacenes generales, arrendadoras financieras, uniones de crédito, sociedades de inversión de capitales, organismos descentralizados o fideicomisos con actividades empresariales, entre otras, siempre que lleven a cabo actividades lucrativas.
  \item Personas morales con fines no lucrativos: Se refiere a aquellas que no persiguen obtener una ganancia económica, por ejemplo: sociedades de inversión, administradoras de fondos para el retiro, sindicatos, cámaras de comercio e industria, colegios de profesionales, instituciones de asistencia o beneficencia y asociaciones civiles sin fines de lucro.
\end{itemize}
%https://www.santanderpyme.com.mx/detalle-noticia/personas-fisicas-o-morales-sabes-a-cual-perteneces.html
  \bigskip
\section{Obligaciones de los contribuyentes} 

Las personas morales y personas físicas que deban presentar declaraciones periódicas o que estén obligadas a expedir facturas electrónicas por los actos o actividades que realicen o por los ingresos que perciban, o que hayan abierto una cuenta a su nombre en las entidades del sistema financiero o en las sociedades cooperativas de ahorro y préstamo, en las que reciban depósitos o realicen operaciones susceptibles de ser sujetas de contribuciones, deben solicitar su inscripción en el Registro Federal de Contribuyentes, proporcionar la información relacionada con su identidad, su domicilio y, en general, sobre su situación fiscal. Asimismo, están obligadas a manifestar al Registro Federal de Contribuyentes su domicilio fiscal. Las personas morales y las personas físicas que deban presentar declaraciones periódicas o que estén obligadas a expedir facturas por los actos o actividades que realicen o por los ingresos que perciban, deben solicitar su firma electrónica. \cite{SAT-OBLIGACIONES}% http://omawww.sat.gob.mx/DerechosyObligaciones/Paginas/obligaciones_contribuyentes.htm
 \bigskip
\section{Programas de capacitación (STPS)}
Un programa de capacitación se define como la descripción detallada de un conjunto de actividades de instrucción-aprendizaje estructuradas de tal forma que conduzcan a alacanzar una serie de objetivos previamente determinados \cite{STPS-PC}.
\\
Las funciones que tiene un programa de capacitación son:
\begin{itemize}
  \item Orientar las actividades de capacitación al señalar los objetivos, actividades, técnicas y recursos que se aplicaran durante el proceso instrucción-aprendizaje.
  \item Seleccionar los contenidos al tener como parámetro el análisis de actividades de manera organizada y sistemática con base en el diagnóstico de necesidades
  \item Ofrecer al instructor la visión de conjunto del evento, permitiéndole conocer la estructura del mismo y auxiliado en la elaboración del plan de sesión
  \item Brindar al capacitando la visión total respecto a cómo será el proceso instrucción-aprendizaje durante el periodo establecido
  \item Proporcionar las bases para efectuar la evaluación del programa, es decir, la forma en que está estructurado respecto a la selección y organización de contenidos y su ubicación en relación al plan de capacitación del cual forma parte 
\end{itemize}
%https://www.gob.mx/cms/uploads/attachment/file/160973/Elaboracion_de_programas_de_capacitaci_n_Anexo_1_250_1.pdf
%%\subsection{} 
%% \newpage

\section{Personas físicas}
Es un individuo que realiza cualquier actividad económica (vendedor, comerciante, empleado, etc.),  tiene obligaciones que cumplir y derechos.
Los régimenes en se que clasifican las Personas Físicas de acuerdo a sus actividades e ingresos son:

\begin{itemize}
  \item  Salarios y en general por la prestación de un servicio personal subordinado
  \item  Actividades Empresariales y Profesionales
  \item  Régimen de Incorporación Fiscal
  \item  Arrendamiento y en general por el uso o goce temporal de bienes inmuebles
  \item Enajenación de Bienes
  \item Adquisición de Bienes
  \item Intereses
  \item Obtención de Premios
  \item Dividendos y en general por las ganancias distribuidas por Personas Morales
\end{itemize}

Los régimenes fiscales \cite{SAT-FISCALES}  que una persona física puede elegir de acuerdo con las actividades empresariales que llevará a cabo son:

\begin{itemize}
  \item Régimen de Incorporación Fiscal: Pueden inscribirse aquellas personas físicas que realicen una actividad comercial o presten algún servicio por los que no requieran título profesional, siempre que sus ingresos anuales no excedan los dos millones de pesos.
  \item Actividad empresarial: pueden tributar aquellas personas físicas que obtienen ingresos por actividades comerciales (restaurantes, cafeterías, escuelas, farmacias, etc.), industriales (minería, textil y calzado, farmacéutica, construcción, etc.).
  \item Actividades Agrícolas, Ganaderas, Silvícolas y Pesqueras (sector primario): Pagarán sus impuestos en este régimen las personas físicas y morales, siempre que sus ingresos por dichas actividades representen cuando menos 90\% de sus ingresos totales.
\end{itemize} %https://www.sat.gob.mx/consulta/09788/emprendedor-conoce-los-regimenes-fiscales


\section{Personas morales}
Una persona moral es una agrupación de individuos que se unen con un fin específico, por ejemplo, una sociedad mercantil o una asociación civil.

Los régimenes fiscales  \cite{SANTANDER-PERSONAS} que una persona moral puede elegir de acuerdo con las actividads empresariales que llevara a cabo son:
\begin{itemize}
  \item Personas morales del régimen general: Se trata de sociedades mercantiles, asociaciones civiles, sociedades cooperativas de producción, instituciones de crédito, se seguros o fianzas; almacenes generales, arrendadoras financieras, uniones de crédito, sociedades de inversión de capitales, organismos descentralizados o fideicomisos con actividades empresariales, entre otras, siempre que lleven a cabo actividades lucrativas.
  \item Personas morales con fines no lucrativos: Se refiere a aquellas que no persiguen obtener una ganancia económica, por ejemplo: sociedades de inversión, administradoras de fondos para el retiro, sindicatos, cámaras de comercio e industria, colegios de profesionales, instituciones de asistencia o beneficencia y asociaciones civiles sin fines de lucro.
\end{itemize}
%https://www.santanderpyme.com.mx/detalle-noticia/personas-fisicas-o-morales-sabes-a-cual-perteneces.html
  \bigskip
\section{Obligaciones de los contribuyentes} 

Las personas morales y personas físicas que deban presentar declaraciones periódicas o que estén obligadas a expedir facturas electrónicas por los actos o actividades que realicen o por los ingresos que perciban, o que hayan abierto una cuenta a su nombre en las entidades del sistema financiero o en las sociedades cooperativas de ahorro y préstamo, en las que reciban depósitos o realicen operaciones susceptibles de ser sujetas de contribuciones, deben solicitar su inscripción en el Registro Federal de Contribuyentes, proporcionar la información relacionada con su identidad, su domicilio y, en general, sobre su situación fiscal. Asimismo, están obligadas a manifestar al Registro Federal de Contribuyentes su domicilio fiscal. Las personas morales y las personas físicas que deban presentar declaraciones periódicas o que estén obligadas a expedir facturas por los actos o actividades que realicen o por los ingresos que perciban, deben solicitar su firma electrónica. \cite{SAT-OBLIGACIONES}% http://omawww.sat.gob.mx/DerechosyObligaciones/Paginas/obligaciones_contribuyentes.htm
 \bigskip
\section{Programas de capacitación (STPS)}
Un programa de capacitación se define como la descripción detallada de un conjunto de actividades de instrucción-aprendizaje estructuradas de tal forma que conduzcan a alacanzar una serie de objetivos previamente determinados \cite{STPS-PC}.
\\
Las funciones que tiene un programa de capacitación son:
\begin{itemize}
  \item Orientar las actividades de capacitación al señalar los objetivos, actividades, técnicas y recursos que se aplicaran durante el proceso instrucción-aprendizaje.
  \item Seleccionar los contenidos al tener como parámetro el análisis de actividades de manera organizada y sistemática con base en el diagnóstico de necesidades
  \item Ofrecer al instructor la visión de conjunto del evento, permitiéndole conocer la estructura del mismo y auxiliado en la elaboración del plan de sesión
  \item Brindar al capacitando la visión total respecto a cómo será el proceso instrucción-aprendizaje durante el periodo establecido
  \item Proporcionar las bases para efectuar la evaluación del programa, es decir, la forma en que está estructurado respecto a la selección y organización de contenidos y su ubicación en relación al plan de capacitación del cual forma parte 
\end{itemize}
%https://www.gob.mx/cms/uploads/attachment/file/160973/Elaboracion_de_programas_de_capacitaci_n_Anexo_1_250_1.pdf
%%\subsection{} 
%%\input{./marcoTeorico/secciones/Empresas}