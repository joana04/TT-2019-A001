
\begin{UseCase}{CU3.01}{Registrar información de la Empresa}{
    Permite al usuario \cdtRef{Actor: Empresa}{Empresa} registrar su información en el sistema, el representante de la  \cdtRef{Actor: Empresa}{Empresa}  presiona el botón \IUbutton{Regístrate} en la sección de Empresas de la pantalla  \IUref{IU1.01}{Índex}, el sistema muestra la pantalla  \IUref{IU3.01}{Registro de Informaci\'on de Empresas} en la cual el usuario podrá ingresar los datos solicitados en la pantalla, después de ingresar sus datos, el usuario podrá presionar el  botón \IUbutton{Cancelar} para cancelar la solicitud o el botón \IUbutton{Siguiente} pararegistrarse en el sistema y continuar con el ingreso de su domicilio en  el caso de uso \IUref{CU3.03}{Registrar ubicación de la Empresa}
    %para ingresar su dirección en la pantalla \IUref{IU3.03}{Registro de Domicilio de Empresa} en la cual el usuario registrará la información de su domicilio, al finalizar el ingreso de datos el usuario podrá  presionar el  botón \IUbutton{Cancelar} para cancelar la solicitud o el botón \IUbutton{Guardar} para concluir su registro en el sistema.    
\bigskip
}
		\UCitem{Versión}{1.0}
		\UCitem {Actor}{\cdtRef{Actor: Empresa}{Empresa}}
		\UCitem{Propósito}{El solicitante podrá registrar sus datos en el sistema.}
		\UCitem{Entradas}{Información ingresada por el usuario:
			\begin{itemize}	
  		        
  		        \item Correo electrónico.
  		        \item Nombre o razón social.
  		        \item Tipo de empresa.
  		        \item Giro empresarial.
  		        \item Número de empleados.
  		        \item Número de becarios necesarios.
				\item Información del representante:
	  		      \begin{itemize}	
	  		        \item Nombre.
	  		        \item Primer apellido.
	  		        \item Segundo apellido.
	  		        \item Télefono.
	  		        \item Teléfono celular. 
				\end{itemize} 
			
			\end{itemize}
	     }
		\UCitem{Salidas}{
			\begin{itemize}
			    \item \IUref{IU3.01}{Registro de Informaci\'on de Empresas}
    			\item \MSGref{MSG1.01}{Registro exitoso.}
    			\item \MSGref{MSG1.03}{Operación cancelada.}
    			\item \MSGref{MSG3.01}{Formato de RFC erróneo.}
    			\item \MSGref{MSG2.04}{Campos obligatorios vacíos.}
    		%	\item \MSGref{MSG2.07}{Formato de Código Postal erróneo.}

    		\end{itemize}}
		\UCitem{Precondiciones}{
    		N/A}
		\UCitem{Postcondiciones}{
		    \begin{itemize}
    		    \item Se genera el registro del usuario en el sistema.
    		 \end{itemize}}
\end{UseCase}
	
	%-------------------------------------- COMIENZA descripción Trayectoria Principal
	\begin{UCtrayectoria}{Registrar información }
	    \UCpaso[\UCactor] Ingresa en el navegador la dirección http://35.224.218.161/TTVinculacionLaboral/index.htm
	    \UCpaso[\UCsist] Muestra la pantalla \IUref{IU1.01}{Índex}.
		\UCpaso[\UCactor] Presiona el botón \IUbutton{Regístrate} en la sección Empresas de la pantalla \IUref{IU1.01}{Índex}.
		\UCpaso[\UCsist] Muestra la pantalla \IUref{IU3.01}{Registro de Informaci\'on de Empresas}. 
		\UCpaso[\UCactor] Ingresa los datos solicitados en la pantalla \IUref{IU3.01}{Registro de Informaci\'on de Empresas}.
		\UCpaso[\UCactor] Presiona el botón \IUbutton{Siguiente}. \Trayref{A-CU3.01}\Trayref{E-CU3.01}
		\UCpaso[\UCsist] Verifica que los campos estén con base en la regla de negocio \hyperlink{RN2}{RN-2 Campo Obligatorio}. \Trayref{B-CU3.01}
		\UCpaso[\UCsist] Valida que el correo no exista en el sistema.\Trayref{D-CU3.01}
			\UCpaso[\UCsist] Verifica que el correo electrónico ingresado esté con base en la regla de negocio \hyperlink{RN13}{Correo electrónico}. \Trayref{G-CU3.01}
		\UCpaso[\UCsist] Verifica que los telefonos esten con base en la regla de negocio \hyperlink{RN10}{Número telefónico (fijo o celular)}. \Trayref{F-CU3.01}
		\UCpaso[\UCsist] Verifica que RFC ingresado por el usuario cumplan con la \hyperlink{RN8}{Formato de Registro Federal de Contribuyentes } . \Trayref{C-CU3.01}
        \UCpaso[\UCsist] Registra la información del solicitante en el sistema.

		\UCpaso[\UCsist] Extiende al caso de uso \IUref{CU3.03}{Registrar ubicación de la Empresa}.

		
	\end{UCtrayectoria}
	

	%---- A Cancelar 
	%---- B Campos vacíos
	%---- C CURP invalido
	%---- D No. de seguro social invalido
	%---- E CP erróneo
	
	
	%---- F formato incorrecto 
	%---- G tamaño incorrecto 

	
	

	%-------------------------------------Trayectorias alternativas A --> Cancelar
	\begin{UCtrayectoriaA}{A-CU3.01}{Presiona el botón \IUbutton{Cancelar} de la pantalla \IUref{IU3.01}{Registro de Informaci\'on de Empresas}}{A}
		\UCpaso[\UCsist] Muestra el mensaje \MSGref{MSG0.03}{Operación cancelada} en la pantalla \IUref{IU1.01}{Índex}.
		\item[- -] - - {\em Fin del caso de uso.} 
	\end{UCtrayectoriaA}

	%-------------------------------------Trayectorias alternativas C --> Campos vacíos
	\begin{UCtrayectoriaA}{B-CU3.01}{Campos vacíos.}{B}
	    \UCpaso[\UCsist]Muestra el mensaje \MSGref{MSG2.04}{Campos obligatorios vacíos}en la pantalla \IUref{IU3.01}{Registro de Informaci\'on de Empresas}.
	    \item[- -] - - {\em Regresa al punto número 5 de la trayectoria principal.}
	\end{UCtrayectoriaA}

    

    %-------------------------------------Trayectorias alternativas E --> RFC erróneo 
	\begin{UCtrayectoriaA}{C-CU3.01}{El RFC no tiene el formato correcto.}{C}
		    %\UCpaso[\UCsist] Muestra la pantalla \IUref{IU2.2}{Ver incidencias}.
			\UCpaso[\UCsist] Muestra el mensaje \MSGref{MSG3.01}{Formato de RFC erróneo} en la pantalla \IUref{IU3.01}{Registro de Informaci\'on de Empresas}.
			\item[- -] - - {\em Regresa al punto número 5 de la trayectoria principal.} 
    \end{UCtrayectoriaA}
    

	    %-------------------------------------Trayectorias alternativas G --> Correo registrado
	\begin{UCtrayectoriaA}{D-CU3.01}{El correo ingresado ya se encuentra registrado en el sistema.}{D}
		\UCpaso[\UCsist] Muestra el mensaje \MSGref{MSG2.08}{Correo ya registrado} en la pantalla \IUref{IU1.01}{Índex}.
		\item[- -] - - {\em Regresa al punto número 5 de la trayectoria principal.} 
	\end{UCtrayectoriaA}	


\begin{UCtrayectoriaA}{E-CU2.01}{ Presiona el link de las Condiciones de uso y  Política de privacidad.}{E}
		\UCpaso[\UCsist] Abre el aviso de privacidad en una pestaña nueva.
		\item[- -] - - {\em Regresa al punto número 6 de la trayectoria principal.} 
	\end{UCtrayectoriaA}
	
		\begin{UCtrayectoriaA}{F-CU3.01}{El teléfono ingresado no tiene el formato correcto.}{F}
		    %\UCpaso[\UCsist] Muestra la pantalla \IUref{IU2.2}{Ver incidencias}.
			\UCpaso[\UCsist] Muestra el mensaje \MSGref{MSG2.10}{Formato de número telefónico incorrecto} en la pantalla \IUref{IU2.01}{Registro de Informaci\'on de Becarios}.
			\item[- -] - - {\em Regresa al punto número 5 de la trayectoria principal.} 
    \end{UCtrayectoriaA}
    
    \begin{UCtrayectoriaA}{G-CU3.01}{El correo electrónico ingresado no tiene el formato correcto.}{G}
		    %\UCpaso[\UCsist] Muestra la pantalla \IUref{IU2.2}{Ver incidencias}.
			\UCpaso[\UCsist] Muestra el mensaje \MSGref{MSG2.11}{Correo electrónico invalido} en la pantalla \IUref{IU2.01}{Registro de Informaci\'on de Becarios}.
			\item[- -] - - {\em Regresa al punto número 5 de la trayectoria principal.} 
    \end{UCtrayectoriaA}

	\subsection{Puntos de Extensión del Caso de Uso}
	
	\begin{UCExtenssionPoint}{Registrar información geográfica}{El usuario desea registrar su información geográfica en el sistema.}{Paso 12 de la trayectoria principal}{Caso de uso \cdtRef{CU3.03}{Registrar ubicación de la Empresa}} 
	\end{UCExtenssionPoint}