\begin{UseCase}{CU2.08}{Editar información geográfica del Becario}{
    Permite al usuario \cdtRef{Actor: Becario}{Becario} editar su información geográfica en el sistema, el usuario presiona el botón  %\IUbutton{Revisar} ubicado en la sección  Ubicación  de la pantalla \IUref{IU2.00}{Inicio de Becarios}, el sistema obtiene la información del usuario y la muestra en la pantalla \IUref{IU2.05}{Visualización de Dirección de Becario}, en esta pantalla el usuario presiona el botón 
    \IUbutton{Editar} en la pantalla \IUref{IU2.05}{Visualización de Dirección de Becario}, el sistema recupera la información del usuario y la muestra en la pantalla \IUref{IU2.08}{Edición de Información geográfica de Becario},  esta pantalla le permite al usuario editar sus datos en el sistema, una vez que ingreso los datos que desea cambiar puede presionar el botón \IUbutton{Cancelar} para cancelar la edición de su información o el botón \IUbutton{Guardar} para almacenar los cambios en el sistema. 
    
    
    \bigskip
}
		\UCitem{Versión}{1.0}
		\UCitem {Actor}{\cdtRef{Actor: Becario}{Becario}}
		\UCitem{Propósito}{El \cdtRef{Actor: Becario}{Becario} podrá editar sus datos geográficos en el sistema.}
		\UCitem{Entradas}{Información ingresada por el usuario:
			\begin{itemize}	
  		        \item Campo o campos que desea editar.
			\end{itemize}
	     }
		\UCitem{Salidas}{
			\begin{itemize}
    			
    			\item \MSGref{MSG1.01}{Registro exitoso.}
    			\item \MSGref{MSG2.07}{Formato de Código Postal erróneo.}
    			\item \MSGref{MSG1.03}{Operación cancelada.}
    			\item \MSGref{MSG2.04}{Campos obligatorios vacíos.}
    			\item \MSGref{MSG2.09}{Formato de Número oficial  erróneo.}
    		\end{itemize}}
		\UCitem{Precondiciones}{
    		Usuario registrado en el sistema.}
		\UCitem{Postcondiciones}{
		    \begin{itemize}
    		    \item Se generan cambios en el registro del usuario en el sistema.
    		 \end{itemize}}
\end{UseCase}
	
	%-------------------------------------- COMIENZA descripción Trayectoria Principal
	\begin{UCtrayectoria}{Editar información }
	    %\UCpaso[\UCactor] Ingresa al sistema con sus credenciales.
	     %\UCpaso[\UCsist] Muestra la pantalla \IUref{IU2.00}{Inicio de Becarios}.
	    %\UCpaso[\UCactor] Presiona el botón \IUbutton{Revisar} de la sección de Ubicación de la pantalla \IUref{IU2.00}{Inicio de Becarios}.
	    %\UCpaso[\UCsist] Obtiene la información del usuario.
		%\UCpaso[\UCsist] Muestra la pantalla \IUref{IU2.05}{Visualización de Dirección de Becario}. 
        \UCpaso[\UCactor]Presiona el botón \IUbutton{Editar} de la pantalla \IUref{IU2.05}{Visualización de Dirección de Becario}.\Trayref{A-CU2.08}
        \UCpaso[\UCsist] Obtiene la información del usuario.
		\UCpaso[\UCsist] Muestra la pantalla \IUref{IU2.08}{Edición de Información geográfica de Becario}. 
		\UCpaso[\UCactor] Ingresa los datos que desee editar de la pantalla \IUref{IU2.08}{Edición de Información geográfica de Becario}.
		\UCpaso[\UCactor] Presiona el botón \IUbutton{Guardar}. \Trayref{D-CU2.08}%antes e
		\UCpaso[\UCsist] Verifica que los campos estén con base en la regla de negocio \hyperlink{RN2}{RN-2 Campo Obligatorio}. \Trayref{B-CU2.08}
		\UCpaso[\UCsist] Verifica que el Código Postal esté con base en la regla de negocio \hyperlink{RN7}{RN-7 C\'odigo Postal}. \Trayref{C-CU2.08}
		\UCpaso[\UCsist] Verifica que el Número oficial (No. exterior)esté con base en la regla de negocio \hyperlink{RN12}{Número oficial(exterior)}. \Trayref{E-CU2.08}
        \UCpaso[\UCsist] Actualiza la información del solicitante en el sistema.
	    \UCpaso[\UCsist] Muestra la pantalla \IUref{IU2.00}{Inicio de Becarios}.
	
	\end{UCtrayectoria}
	

	%---- A Cerrar
	%---- B Campos vacíos
	%---- C CP invalido

	%---- D-E Cancelar
	

	
	

	%-------------------------------------Trayectorias alternativas A --> Cerrar
	\begin{UCtrayectoriaA}{A-CU2.08}{Presiona el botón \IUbutton{Cerrar} de la pantalla \IUref{IU2.05}{Visualización de Dirección de Becario}}{A}
		\UCpaso[\UCsist] Muestra la pantalla \IUref{IU2.00}{Inicio de Becarios}.
		\item[- -] - - {\em Fin del caso de uso.} 
	\end{UCtrayectoriaA}

	%-------------------------------------Trayectorias alternativas b --> Campos vacíos
	\begin{UCtrayectoriaA}{B-CU2.08}{Campos vacíos.}{B}
	    \UCpaso[\UCsist]Muestra el mensaje \MSGref{MSG2.04}{Campos obligatorios vacíos}en la pantalla \IUref{IU2.08}{Edición de Información geográfica de Becario}.
	    \item[- -] - - {\em Regresa al punto número 4 de la trayectoria principal.}
	\end{UCtrayectoriaA}
	
	%----------------------------------------D CP invalido
		\begin{UCtrayectoriaA}{C-CU2.08}{El Código postal no tiene el formato correcto.}{C}

			\UCpaso[\UCsist] Muestra el mensaje \MSGref{MSG2.07}{Formato de Código Postal erróneo.} en la pantalla \IUref{IU2.08}{Edición de Información geográfica de Becario}.
			\item[- -] - - {\em Regresa al punto número 4 de la trayectoria principal.} 
    \end{UCtrayectoriaA}

    
	 %-------------------------------------Trayectorias alternativas E--> Cancelar
	\begin{UCtrayectoriaA}{D-CU2.08}{Presiona el botón \IUbutton{Cancelar} de la pantalla \IUref{IU2.08}{Edición de Información geográfica de Becario}}{D}
		\UCpaso[\UCsist] Muestra el mensaje \MSGref{MSG0.03}{Operación cancelada} en la pantalla \IUref{IU2.00}{Inicio de Becarios}.
		\item[- -] - - {\em Fin del caso de uso.} 
	\end{UCtrayectoriaA}
    
    
    \begin{UCtrayectoriaA}{E-CU2.08}{El No. exterior no tiene el formato correcto.}{E}

			\UCpaso[\UCsist] Muestra el mensaje \MSGref{MSG2.09}{Formato de Número oficial  erróneo} en la pantalla \IUref{IU2.08}{Edición de Información geográfica de Becario}.
			\item[- -] - - {\em Regresa al punto número 4 de la trayectoria principal.} 
    \end{UCtrayectoriaA}
  %-------------------PUNTOS DE EXTENSION----------------  

