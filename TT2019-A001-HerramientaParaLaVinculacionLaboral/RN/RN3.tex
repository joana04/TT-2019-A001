%========================================================
% Regla del Negocio #3 - Formato de CURP
%-----------------------------------------------
\section{Regla de Negocio 3.- Formato de CURPs}

\begin{BussinesRule}{RN3}{Formato de CURP}
	\BRitem[Tipo:] Regla estructural. 
				% Otras opciones para tipo: 
				% - Regla de integridad referencial o estructural. 
				% - Regla de operación, (calcular o determinar un valor.).
				% - Regla de inferencia de un hecho.
	\BRitem[Clase:] Habilitadora. 
				% Otras opciones para clase: Habilitadora, Cronometrada, Ejecutive.
	\BRitem[Nivel:] Control. % Otras opciones para nivel: Control, Influencia.
	\BRitem[Descripción:] La Clave Única de Registro de Población (CURP) es un instrumento de registro que se asigna a todas las personas que viven en el territorio nacional, así como a los mexicanos que residen en el extranjero\cite{SG}. La instancia responsable de asignar la CURP y de expedir la Constancia respectiva es el Registro Nacional de Población.\\
	El CURP está integrado con dieciocho elementos, representados por letras y números, que se generan a partir de los datos contenidos en el documento probatorio de tu identidad como acta de nacimiento, carta de naturalización o documento migratorio \cite{SG}, y que se refieren a:
	        \begin{itemize}
	            \item El primer apellido, segundo apellido y nombre de pila.
	            \item La fecha de nacimiento.
	            \item El sexo.
	            \item La entidad federativa de nacimiento.
	            \item Los dos últimos elementos de la CURP evitan la duplicidad de la Clave y garantizan su correcta integración.
	           
	        \end{itemize}
	        
	
	\BRitem[Motivación:] Registrar en el sistema CURPs con estructuras correctas.
	\BRitem[Sentencia:] $\forall\ x \in CURP \Rightarrow  [A-Z]{4}[0-9]{6}[H|M][A-Z]{5}[0-9]{2}$
	\BRitem[Ejemplo positivo:] Cumplen la regla de negocio los siguientes CURPs:
        \begin{itemize}
			\item OORE970504MMCSDS03
			\item OORI061120MMCSDS19
        \end{itemize}
	\BRitem[Ejemplo negativo:] No cumplen la regla de negocio los siguientes CURPs.
		\begin{itemize}
        	\item OORE9705MMCSDS03
			\item OORI061120M82SDSKL
        	
    \end{itemize}
\end{BussinesRule}




