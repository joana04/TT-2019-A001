%========================================================
% Regla del Negocio #8 - Formato de RFC
%-----------------------------------------------
\section{Regla de Negocio 8.- Formato de Registro Federal de Contribuyentes}

\begin{BussinesRule}{RN8}{Formato de Registro Federal de Contribuyentes }
	\BRitem[Tipo:] Regla estructural. 
				% Otras opciones para tipo: 
				% - Regla de integridad referencial o estructural. 
				% - Regla de operación, (calcular o determinar un valor.).
				% - Regla de inferencia de un hecho.
	\BRitem[Clase:] Habilitadora. 
				% Otras opciones para clase: Habilitadora, Cronometrada, Ejecutiva.
	\BRitem[Nivel:] Control. % Otras opciones para nivel: Control, Influencia.
	\BRitem[Descripción:]El RFC es una clave que identifica como contribuyentes a las personas físicas o morales en México para controlar el pago de impuestos frente al SAT, el Servicio de Administración Tributaria \cite{RFC}. Su denominación abreviada será RFC y esiten dos tipos, el RC para personas físicas y para personas morales o empresas.
	\begin{itemize}
	            \item Personas físicas:
	            \begin{itemize}
	                \item Primera letra y primera vocal interna del primer apellido
                    \item Primera letra del segundo apellido
                    \item Primera letra del primer nombre del contribuyente
                    \item Fecha de nacimiento en formato aa/mm/dd
                    \item Homoclave calculada con un algoritmo de público conocimiento, con dígito verificador para evitar repeticiones (asignado por el SAT)
	            \end{itemize}
	            \item Personas morales:
	            \begin{itemize}
	                \item Si el nombre de la empresa tiene tres palabras, usar la primera letra de cada una. Si tiene dos, usar la primera de la primer palabra, y las primeras dos de la segunda. Si tiene una, usar sus primeras tres letras.
                \item Fecha de constitución de la empresa en formato aa/mm/dd
                \item Homoclave (asignada por el SAT)
	            \end{itemize}
	        \end{itemize}

	\BRitem[Motivación:] Registrar en el sistema RFCs con estructuras correctas.
	\BRitem[Sentencia:]  $\forall\ x \in RFC \Rightarrow   ([A-Z,\tilde{N},\&]{3,4}([0-9]{2})(0[1-9]|1[0-2])(0[1-9]|1[0-9]|2[0-9]|3[0-1])[A-Z0-9]{3})$
	\BRitem[Ejemplo positivo:] Cumplen la regla de negocio los siguientes RFCs:
        \begin{itemize}
			\item GOAP780710RH7
			\item IAT030120E60
        \end{itemize}
	\BRitem[Ejemplo negativo:] No cumplen la regla de negocio los siguientes RFCs.
		\begin{itemize}
        	\item G4Pu345670GGT
			\item IA6030150E60
        	
    \end{itemize}
\end{BussinesRule}



