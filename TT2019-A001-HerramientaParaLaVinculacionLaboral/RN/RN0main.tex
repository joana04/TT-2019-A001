%=========================================================
\chapter{Reglas del Negocio}
\label{cap:reglasNegocio}

Las reglas de negocio son directivas que tienen como fundamento la misión de un negocio y como objetivo regir estrategias para poder conseguir esta misión. Una regla de negocio no requiere de una interpretación adicional. Estas reglas también son una fuente de información muy relevante ya que generalmente establecen las relaciones entre dos o más términos del negocio.

En el documento se presentarán estas reglas en forma de secciones indicando los siguientes atributos:

\begin{itemize}
    \item \textbf{Id:} Es el identificador de la regla de negocio con la cual se podrá referenciar a lo largo del documento.
    \item \textbf{Nombre:} Indica el nombre de la regla de negocio el cual debe describir de forma concisa en qué consiste la regla.
    \item \textbf{Tipo:} Indica el tipo de regla de negocio de acuerdo a como se aplica.
        \begin{itemize}
            \item \textbf{Habilitadora:} Permite realizar el proceso en el que la regla se ve involucrada.
            \item \textbf{Cronometrada:} Recibe parámetros y con respecto a eso realiza el proceso.
            \item \textbf{Ejecutiva:} Es aquella que se debe llevar a cabo cuando una autoridad se ve involucrada para que el proceso concluya.
        \end{itemize}
    \item \textbf{Clase:} Indica la naturaleza de la regla de negocio.
        \begin{itemize}
            \item \textbf{Condición:} Es una regla que cumple una condición para llevarse a cabo.
            \item \textbf{Integridad:} Es una regla que indica validaciones que, de no ser tomadas en cuenta se pone en peligro la integridad de la informacion.
            \item \textbf{Autorización:} Son restricciones en las que se ven involucradas palabras como, al menos uno.
        \end{itemize} 
    \item \textbf{Nivel:} Indica cómo es que la regla se toma en cuenta para el desarrollo del sistema.
        \begin{itemize}
            \item \textbf{Controla:} Define que el sistema se encargará de vigilar el cumplimiento de la regla en todo momento.
            \item \textbf{Influencia:} Sugiere formas en las que se debe realizar la operación, pero no la limita. De tal forma que el sistema dará facilidades para evitar esas situaciones o advertirá cada vez que se detecte que la regla no es tomada en cuenta.
        \end{itemize}
    \item \textbf{Descripción:} Es un pequeño resumen que ayuda a entender la regla de negocio.
    \item \textbf{Motivación:} La razón detrás de la existencia de la regla de negocio.
    \item \textbf{Sentencia:} Descripción formal o matemática de la regla de negocio.
\end{itemize}


%%Aqui van los inputs a las reglas de negocio

%   Regla de negocio Mario
%   - RFC
%   - CURP (Personas) 

%
% Regla del Negocio #1 - Perfil de becario
%-----------------------------------------------
\section{Regla de Negocio 1.- Perfil de Becario}

\begin{BussinesRule}{RN1}{Perfil de Becario}
	\BRitem[Tipo:] Regla de restricción (Validación).
				% Otras opciones para tipo: 
				% - Regla de integridad referencial o estructural. 
				% - Regla de operación, (calcular o determinar un valor.).
				% - Regla de inferencia de un hecho.
	\BRitem[Clase:] Habilitadora. 
				% Otras opciones para clase: Habilitadora, Cronometrada, Ejecutiva.
	\BRitem[Nivel:] Control. % Otras opciones para nivel: Control, Influencia.
	\BRitem[Descripción:] Un becario es aquel usuario registrado en el sistema que no estudia ni trabaja.
	\BRitem[Motivación:] Tener únicamente registrados a las personas que cumplan con:
	\begin{itemize}
	    \item No estudiar.
	    \item No trabajar.
	\end{itemize}
	
\end{BussinesRule}



%========================================================
%========================================================
% Regla del Negocio #2 - Campo obligatorio
%-----------------------------------------------
\section{Regla de Negocio 2.- Campo obligatorio}

\begin{BussinesRule}{RN2}{Campo obligatorio}
	\BRitem[Tipo:] Regla de restricción (Validación).
				% Otras opciones para tipo: 
				% - Regla de integridad referencial o estructural. 
				% - Regla de operación, (calcular o determinar un valor.).
				% - Regla de inferencia de un hecho.
	\BRitem[Clase:] Habilitadora. 
				% Otras opciones para clase: Habilitadora, Cronometrada, Ejecutiva.
	\BRitem[Nivel:] Control. % Otras opciones para nivel: Control, Influencia.
	\BRitem[Descripción:] Los campos de este formulario que contienen al inicio un * no pueden ser un campo vacío.
	\BRitem[Motivación:] Tener los datos correspondientes para validar el campo en el sistema. 
\end{BussinesRule}

%========================================================
% Regla del Negocio #3 - Formato de CURP
%-----------------------------------------------
\section{Regla de Negocio 3.- Formato de CURPs}

\begin{BussinesRule}{RN3}{Formato de CURP}
	\BRitem[Tipo:] Regla estructural. 
				% Otras opciones para tipo: 
				% - Regla de integridad referencial o estructural. 
				% - Regla de operación, (calcular o determinar un valor.).
				% - Regla de inferencia de un hecho.
	\BRitem[Clase:] Habilitadora. 
				% Otras opciones para clase: Habilitadora, Cronometrada, Ejecutive.
	\BRitem[Nivel:] Control. % Otras opciones para nivel: Control, Influencia.
	\BRitem[Descripción:] La Clave Única de Registro de Población (CURP) es un instrumento de registro que se asigna a todas las personas que viven en el territorio nacional, así como a los mexicanos que residen en el extranjero\cite{SG}. La instancia responsable de asignar la CURP y de expedir la Constancia respectiva es el Registro Nacional de Población.\\
	El CURP está integrado con dieciocho elementos, representados por letras y números, que se generan a partir de los datos contenidos en el documento probatorio de tu identidad como acta de nacimiento, carta de naturalización o documento migratorio \cite{SG}, y que se refieren a:
	        \begin{itemize}
	            \item El primer apellido, segundo apellido y nombre de pila.
	            \item La fecha de nacimiento.
	            \item El sexo.
	            \item La entidad federativa de nacimiento.
	            \item Los dos últimos elementos de la CURP evitan la duplicidad de la Clave y garantizan su correcta integración.
	           
	        \end{itemize}
	        
	
	\BRitem[Motivación:] Registrar en el sistema CURPs con estructuras correctas.
	\BRitem[Sentencia:] $\forall\ x \in CURP \Rightarrow  [A-Z]{4}[0-9]{6}[H|M][A-Z]{5}[0-9]{2}$
	\BRitem[Ejemplo positivo:] Cumplen la regla de negocio los siguientes CURPs:
        \begin{itemize}
			\item OORE970504MMCSDS03
			\item OORI061120MMCSDS19
        \end{itemize}
	\BRitem[Ejemplo negativo:] No cumplen la regla de negocio los siguientes CURPs.
		\begin{itemize}
        	\item OORE9705MMCSDS03
			\item OORI061120M82SDSKL
        	
    \end{itemize}
\end{BussinesRule}





%========================================================
% Regla del Negocio #4 - Formato de NSS
%-----------------------------------------------
\section{Regla de Negocio 3.- Formato de Número de Seguro Social }

\begin{BussinesRule}{RN3}{Formato de Número de Seguro Social (NSS)}
	\BRitem[Tipo:] Regla estructural. 
				% Otras opciones para tipo: 
				% - Regla de integridad referencial o estructural. 
				% - Regla de operación, (calcular o determinar un valor.).
				% - Regla de inferencia de un hecho.
	\BRitem[Clase:] Habilitadora. 
				% Otras opciones para clase: Habilitadora, Cronometrada, Ejecutiva.
	\BRitem[Nivel:] Control. % Otras opciones para nivel: Control, Influencia.
	\BRitem[Descripción:]El Número de Seguro Social (NSS) es un número compuesto de once (11) dígitos con el cual los trabajadores que cotizan en el IMSS son identificados por dicha institución desde el mismo momento de su afiliación. Es de carácter personal, permanente y único \cite{NSS}. Su secuencia de números se compone de los siguientes elementos:
	        \begin{itemize}
	            \item Los dos primeros dígitos Refiere a la subdelegación en el que fue afiliado
	            \item Tercer y cuarto digito corresponden al año de afiliación
	            \item Quinto y sexto digito se refieren a la fecha de nacimiento del afiliado.
	            \item Los siguientes cuatro dígitos son los que el IMSS le asignado al trabajador.
	            \item El último digito es el número de verificación del trabajador ante el IMSS.
	           
	        \end{itemize}
	        
	
	\BRitem[Motivación:] Registrar en el sistema NSSs con estructuras correctas.
	\BRitem[Sentencia:] $\forall\ x \in NSS \Rightarrow     [0-9]{11}$
	\BRitem[Ejemplo positivo:] Cumplen la regla de negocio los siguientes NSSs:
        \begin{itemize}
			\item 14567892345
			\item 98547928354
        \end{itemize}
	\BRitem[Ejemplo negativo:] No cumplen la regla de negocio los siguientes NSSs.
		\begin{itemize}
        	\item 8654935
			\item 67435GTJ2
        	
    \end{itemize}
\end{BussinesRule}


%========================================================
% Regla del Negocio #5 - Adjuntar un archivo 
%-----------------------------------------------
\section{Regla de Negocio 5.- Adjuntar un archivo válido}

\begin{BussinesRule}{RN5}{Adjuntar un archivo válido}
%\label{BR:RN5}
	\BRitem[Tipo:] Regla de integridad. 
				% Otras opciones para tipo: 
				% - Regla de integridad referencial o estructural. 
				% - Regla de operación, (calcular o determinar un valor.).
				% - Regla de inferencia de un hecho.
	\BRitem[Clase:] Habilitadora. 
				% Otras opciones para clase: Habilitadora, Cronometrada, Ejecutive.
	\BRitem[Nivel:] Control. % Otras opciones para nivel: Control, Influencia.
	\BRitem[Descripción:] Al adjuntar un archivo este solo será recibido por el sistema si está en el formato PDF.
	\BRitem[Motivación:] Evitar que se guarden en el sistema archivos que no requiera el sistema para el registro de becarios.
	\BRitem[Sentencia:] 
	   Se tiene a un becario que desea adjuntar sus documentos para registrarse en el sistema.
	\BRitem[Ejemplo positivo:] El usuario adjunta su archivo de CURP ``CURP.pdf''.
	
	\BRitem[Ejemplo negativo:] El usuario adjunta su archivo de CURP ``CURP.xml''. 
\end{BussinesRule}
%========================================================
% Regla del Negocio #6 - Tamaño de archivo incorrecto
%-----------------------------------------------
\section{Regla de Negocio 6.- Tamaño de archivo incorrecto}
\begin{BussinesRule}{RN6}{Tamaño de archivo incorrecto}

	\BRitem[Tipo:] Regla de integridad referencial o estructural. 
				% Otras opciones para tipo: 
				% - Regla de integridad referencial o estructural. 
				% - Regla de operación, (calcular o determinar un valor.).
				% - Regla de inferencia de un hecho.
	\BRitem[Clase:] Habilitadora. 
				% Otras opciones para clase: Habilitadora, Cronometrada, Ejecutive.
	\BRitem[Nivel:] Control. % Otras opciones para nivel: Control, Influencia.
	\BRitem[Descripción:]	Los archivos adjuntos deben tener un tamaño límite de 5 MB. 
	\BRitem[Motivación:] Evitar archivos de peso excesivo que puedan alentar el funcionamiento del sistema.
	\BRitem[Sentencia:]
	    Sea x un archivo.jpg y n el tamaño que puede tener el archivo x, entonces n es menor o igual a 3 MB.

	\BRitem[Ejemplo positivo:] Archivo con tamaño dentro del rango permitido:		
        \begin{itemize}
        	\item Archivo.jpg (3 MB)
			\item Archivo2.jpg (1.5 MB)
        \end{itemize}
	
	\BRitem[Ejemplo negativo:] Archivo con tamaño excedente del rango permitido:
		\begin{itemize}
        	\item Archivo.jpg (176 MB)
			\item Archivo2.jpg (20 MB)
        \end{itemize}
	%\BRitem[Referenciado por:] \hyperlink{CUCE3.2}{CUCE3.2}, \hyperlink{CUCE3.3}{CUCE3.3}.
\end{BussinesRule}


%========================================================
% Regla del Negocio #7 - Formato de CP
%-----------------------------------------------
\section{Regla de Negocio 7.- Formato de Código Postal}

\begin{BussinesRule}{RN7}{Formato de Código Postal}
	\BRitem[Tipo:] Regla estructural. 
				% Otras opciones para tipo: 
				% - Regla de integridad referencial o estructural. 
				% - Regla de operación, (calcular o determinar un valor.).
				% - Regla de inferencia de un hecho.
	\BRitem[Clase:] Habilitadora. 
				% Otras opciones para clase: Habilitadora, Cronometrada, Ejecutiva.
	\BRitem[Nivel:] Control. % Otras opciones para nivel: Control, Influencia.
	\BRitem[Descripción:]Clave numérica compuesta por cinco dígitos que identifica y ubica un área geográfica del país y la oficina postal que la sirve, para facilitar al correo, el encaminamiento, la distribución y el reparto de la materia postal \cite{CP}. Su denominación abreviada será C.P.

	\BRitem[Motivación:] Registrar en el sistema CPs con estructuras correctas.
	\BRitem[Sentencia:] $\forall\ x \in CP \Rightarrow   [0-9]{1}[1-9]{1}[0-9]{3}$
	\BRitem[Ejemplo positivo:] Cumplen la regla de negocio los siguientes CPs:
        \begin{itemize}
			\item 57630
			\item 45678
        \end{itemize}
	\BRitem[Ejemplo negativo:] No cumplen la regla de negocio los siguientes CP:
		\begin{itemize}
        	\item 00564
			\item O54t6
        	
    \end{itemize}
\end{BussinesRule}




%========================================================
% Regla del Negocio #8 - Formato de RFC
%-----------------------------------------------
\section{Regla de Negocio 8.- Formato de Registro Federal de Contribuyentes}

\begin{BussinesRule}{RN8}{Formato de Registro Federal de Contribuyentes }
	\BRitem[Tipo:] Regla estructural. 
				% Otras opciones para tipo: 
				% - Regla de integridad referencial o estructural. 
				% - Regla de operación, (calcular o determinar un valor.).
				% - Regla de inferencia de un hecho.
	\BRitem[Clase:] Habilitadora. 
				% Otras opciones para clase: Habilitadora, Cronometrada, Ejecutiva.
	\BRitem[Nivel:] Control. % Otras opciones para nivel: Control, Influencia.
	\BRitem[Descripción:]El RFC es una clave que identifica como contribuyentes a las personas físicas o morales en México para controlar el pago de impuestos frente al SAT, el Servicio de Administración Tributaria \cite{RFC}. Su denominación abreviada será RFC y esiten dos tipos, el RC para personas físicas y para personas morales o empresas.
	\begin{itemize}
	            \item Personas físicas:
	            \begin{itemize}
	                \item Primera letra y primera vocal interna del primer apellido
                    \item Primera letra del segundo apellido
                    \item Primera letra del primer nombre del contribuyente
                    \item Fecha de nacimiento en formato aa/mm/dd
                    \item Homoclave calculada con un algoritmo de público conocimiento, con dígito verificador para evitar repeticiones (asignado por el SAT)
	            \end{itemize}
	            \item Personas morales:
	            \begin{itemize}
	                \item Si el nombre de la empresa tiene tres palabras, usar la primera letra de cada una. Si tiene dos, usar la primera de la primer palabra, y las primeras dos de la segunda. Si tiene una, usar sus primeras tres letras.
                \item Fecha de constitución de la empresa en formato aa/mm/dd
                \item Homoclave (asignada por el SAT)
	            \end{itemize}
	        \end{itemize}

	\BRitem[Motivación:] Registrar en el sistema RFCs con estructuras correctas.
	\BRitem[Sentencia:]  $\forall\ x \in RFC \Rightarrow   ([A-Z,\tilde{N},\&]{3,4}([0-9]{2})(0[1-9]|1[0-2])(0[1-9]|1[0-9]|2[0-9]|3[0-1])[A-Z0-9]{3})$
	\BRitem[Ejemplo positivo:] Cumplen la regla de negocio los siguientes RFCs:
        \begin{itemize}
			\item GOAP780710RH7
			\item IAT030120E60
        \end{itemize}
	\BRitem[Ejemplo negativo:] No cumplen la regla de negocio los siguientes RFCs.
		\begin{itemize}
        	\item G4Pu345670GGT
			\item IA6030150E60
        	
    \end{itemize}
\end{BussinesRule}




%========================================================
%========================================================
% Regla del Negocio #9 - Contraseña
%-----------------------------------------------
\section{Regla de Negocio 9.- Contraseña valida}

\begin{BussinesRule}{RN9}{Contraseña valida}
	\BRitem[Tipo:] Regla de restricción (Validación).
				% Otras opciones para tipo: 
				% - Regla de integridad referencial o estructural. 
				% - Regla de operación, (calcular o determinar un valor.).
				% - Regla de inferencia de un hecho.
	\BRitem[Clase:] Habilitadora. 
				% Otras opciones para clase: Habilitadora, Cronometrada, Ejecutiva.
	\BRitem[Nivel:] Control. % Otras opciones para nivel: Control, Influencia.
	\BRitem[Descripción:] La contraseña debera cumplir con tener una longitud mínima de 8 caracteres y máxima de 16 caracteres, contener una letra mayúscula, una minúscula, un número y un caracter especial \cite{contra}.
	\BRitem[Motivación:] Almacenar constraseñas seguras para los usuarios. 
	\BRitem[Sentencia:] $\forall\ x \in Contraseña \Rightarrow     {(?=.+[0-9])(?=.+[a-z])(?=.+[A-Z])(?=.+[\#\$\&\%@])} {8-16}$
	
		\BRitem[Ejemplo positivo:] Contraseña valida:		
        \begin{itemize}
        	\item P4s\$word
        \end{itemize}
	
	\BRitem[Ejemplo negativo:] Contraseña invaida:
		\begin{itemize}
        	\item Contraseña   
			\end{itemize}
\end{BussinesRule}

%========================================================
%========================================================
% Regla del Negocio #10 - Número de becarios
%-----------------------------------------------
\section{Regla de Negocio 10.- Número telefónico (fijo o celular)}

\begin{BussinesRule}{RN10}{Número telefónico (fijo o celular)}
	\BRitem[Tipo:] Regla de restricción (Validación).
				% Otras opciones para tipo: 
				% - Regla de integridad referencial o estructural. 
				% - Regla de operación, (calcular o determinar un valor.).
				% - Regla de inferencia de un hecho.
	\BRitem[Clase:] Habilitadora. 
				% Otras opciones para clase: Habilitadora, Cronometrada, Ejecutiva.
	\BRitem[Nivel:] Control. % Otras opciones para nivel: Control, Influencia.
	\BRitem[Descripción:] El teléfono ingresado por el usuario puede ser fijo o celular.
	\BRitem[Motivación:] Que los becarios ingresen números telefónicos validos. 
	
		\BRitem[Sentencia:] $\forall\ x \in N\'umero Tel\'efonico \Rightarrow     [0-9]{10}$
	\BRitem[Ejemplo positivo:] Cumplen la regla de negocio los siguientes Teléfonos:
        \begin{itemize}
			\item 5544678932
			\item 5432146789
        \end{itemize}
	\BRitem[Ejemplo negativo:] No cumplen la regla de negocio los siguientes Teléfonos.
		\begin{itemize}
        	\item 34875116
			\item 67435GTJ2
        	
    \end{itemize}

\end{BussinesRule}

%========================================================
%========================================================
% Regla del Negocio #11 - Datos que se pueden editar
%-----------------------------------------------
\section{Regla de Negocio 11.- Datos que se pueden editar}

\begin{BussinesRule}{RN11}{Datos que se pueden editar}
	\BRitem[Tipo:] Regla de restricción (Validación).
				% Otras opciones para tipo: 
				% - Regla de integridad referencial o estructural. 
				% - Regla de operación, (calcular o determinar un valor.).
				% - Regla de inferencia de un hecho.
	\BRitem[Clase:] Habilitadora. 
				% Otras opciones para clase: Habilitadora, Cronometrada, Ejecutiva.
	\BRitem[Nivel:] Control. % Otras opciones para nivel: Control, Influencia.
	\BRitem[Descripción:] El usuario puede editar solamente ciertos datos. En caso de que quiera editar algun dato que no se encuentre en la lista siguiente, debera acudir con el administrador del sistema.\\
	Datos que se pueden editar:
	\begin{itemize}
	    \item Becario:
	        \begin{itemize}
	            \item Télefono fijo.
	            \item Télefono celular.
	            \item Discapacidad.
	            \item Grado de estudios.
	            \item Dirección.
	            \item Documentos.
	        \end{itemize}
	   \item Empresa:
	        \begin{itemize}
	            \item Giro empresarial.
	            \item Información personal o del representante de la empresa.
	        \end{itemize}
	\end{itemize}
	\BRitem[Motivación:] Que el usuario pueda editar inormación que no pueda tener un impacto en el sistema. 

\end{BussinesRule}

%========================================================
%========================================================
% Regla del Negocio #12 - Número oficia-exterior
%-----------------------------------------------
\section{Regla de Negocio 12.- Número oficial(exterior)}

\begin{BussinesRule}{RN12}{Número oficial(exterior)}
	\BRitem[Tipo:] Regla de restricción (Validación).
				% Otras opciones para tipo: 
				% - Regla de integridad referencial o estructural. 
				% - Regla de operación, (calcular o determinar un valor.).
				% - Regla de inferencia de un hecho.
	\BRitem[Clase:] Habilitadora. 
				% Otras opciones para clase: Habilitadora, Cronometrada, Ejecutiva.
	\BRitem[Nivel:] Control. % Otras opciones para nivel: Control, Influencia.
	\BRitem[Descripción:]  El Número Oficial es la asignación alfanumérica que le corresponde a un predio en la secuencia predeterminada por cada vía pública para su correcta identificación \cite{NO}. Longitud máxima 10 dígitos. 
	\BRitem[Motivación:] Almacenar números exteriores validos. 

\end{BussinesRule}

%========================================================
%========================================================
% Regla del Negocio #12 - Número oficia-exterior
%-----------------------------------------------
\section{Regla de Negocio 13.- Correo electrónico}

\begin{BussinesRule}{RN13}{Correo electrónico}
	\BRitem[Tipo:] Regla de restricción (Validación).
				% Otras opciones para tipo: 
				% - Regla de integridad referencial o estructural. 
				% - Regla de operación, (calcular o determinar un valor.).
				% - Regla de inferencia de un hecho.
	\BRitem[Clase:] Habilitadora. 
				% Otras opciones para clase: Habilitadora, Cronometrada, Ejecutiva.
	\BRitem[Nivel:] Control. % Otras opciones para nivel: Control, Influencia.
	\BRitem[Descripción:]  Servicio que permite el intercambio de mensajes a través de sistemas de comunicación electrónicos. Longitud máxima 60 caracteres. 
	\BRitem[Motivación:] Almacenar correos electrónicos validos.
	
	\BRitem[Sentencia:] $\forall\ x \in Correo Electr\'onico \Rightarrow     [a-zA-Z0-9_]+([.][a-zA-Z0-9_]+)*@[a-zA-Z0-9_]+([.][a-zA-Z0-9_]+)*[.][a-zA-Z]{1,5}$
	\BRitem[Ejemplo positivo:] Cumplen la regla de negocio los siguientes correos electrónicos:
        \begin{itemize}
			\item email.123@dominio.com
        \end{itemize}
	\BRitem[Ejemplo negativo:] No cumplen la regla de negocio los siguientes correos electrónicos.
		\begin{itemize}
        	\item 34875116dominio.com
        	
    \end{itemize}


\end{BussinesRule}

%\input{RN/RN25.tex}




